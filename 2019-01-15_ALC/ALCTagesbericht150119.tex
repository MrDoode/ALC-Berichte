\documentclass[11pt, a4paper]{article}
\usepackage{fullpage}
\usepackage{mathtools}

\begin{document}

\title{ALC Tagesbericht\\ vom 15.01.19}
\author{Janosch Ehlers, Jacqueline Preis}
\maketitle

	\begin{center}
	\textsc{Elektrochemische Experimente}
	\end{center}

\section{9.1 Fällungs- /Redoxreihe der Metalle}

In diesem Versuch werden verschiedene Redoxreaktionen betrachtet, um die beteiligten Metalle, anhand der Ergebnisse, in die Redoxreihe einzuordnen.


\begin{center}
\textsc{Teilversuch A}
\end{center}

\textsc{Versuchsdurchführung:} Siehe Skript.\\

\textsc{Beobachtung:}\hspace{5mm} Die Ergebnisse können folgender Tabelle entnommen werden:\\
\begin{center}
\textsc{Tabelle 1: Ergebnisse Versuch 9.1}\\
\begin{tabular}{ccccc}
 & & $ZnSO_4$-Lösung & $FeSO_4$-Lösung & $AgNO_3$-Lösung\\
 Fe & + & - & - & +\\
 Zn & + & - & + & +\\
 Cu & - & - & - & +\\
\end{tabular}
\end{center}
+: reagiert
-: reagiert nicht
Beobachtungen:
Kupfernitrat-Lösung und Eisen: Auf dem Eisennagel bildet sich eine orange Kupferschicht.\\
Kupfernitrat-Lösung und Zink: Die Zinkspäne färben sich leicht schwarz.\\
Kupfernitrat-Lösung und Kupfer: Man kann keine Veränderungen beobachten. \\

Zinksulfat-Lösung und Eisen: Man kann keine Veränderungen beobachten.\\
Zinksulfat-Lösung und Zink: Man kann keine Veränderungen beobachten.\\
Zinksulfat-Lösung und Kupfer: Man kann keine Veränderungen beobachten.\\

Eisensulfat-Lösung und Eisen: Man kann keine Veränderungen beobachten.\\
Eisensulfat-Lösung und Zink: Auf den Zinkspänen setzt sich eine Eisenschicht ab. \\
Eisensulfat-Lösung und Kupfer: Man kann keine Veränderungen beobachten.\\

Silbernitrat-Lösung und Eisen: Auf dem Eisennagel setzt sich eine Silberschicht ab. \\
Silbernitrat-Lösung und Zink: Die Zinkspäne färben sich schwarz. \\
Silbernitrat-Lösung und Kupfer: Das Kupferblech färbt sich schwarz/silbrig. \\

\textsc{Auswertung:}\hspace{8mm} Reaktionsgleichungen:
Eisen und Kupfernitrat-Lösung: $Fe_{(s)} + Cu(NO_3)_2 \rightleftharpoons Fe(NO_3)_2 + Cu_{(s)}$\\
Zink und Kupfernitrat-Lösung: $Zn_{(s)} + Cu(NO_3)_2 \rightleftharpoons Zn(NO_3)_2 + Cu_{(s)}$\\
Zink und Eisensulfat: $Zn_{(s)} + FeSO_4 \rightleftharpoons ZnSO_4 + Fe_{(s)}$\\
Eisen und Silbernitrat-Lösung: $Fe_{(s)} + 2AgNO_3 \rightleftharpoons Fe(NO_3)_2 + 2Ag_{(s)}$\\
Zink und Silbernitrat: $Zn_{(s)} + 2AgNO_3 \rightleftharpoons Zn(NO_3)_2 + 2Ag$\\
Kupfer und Silbernitrat: $Cu_{(s)} + 2AgNO_3 \rightleftharpoons Cu(NO_3)_2 + 2Ag_{(s)}$\\führung: Siehe Skript.\\

\begin{center}
\textsc{Teilversuch B}
\end{center}

\textsc{Beobachtung:}\hspace{5mm} Das Kupfer bildet eine Suspension in der Salzsäure. Das Zinn hingegen bildet Klumpen das Eisen Pulver geht ebenfalls ins Suspension.
\newpage
\section{9.2 Daniell Element}

\textsc{Versuchsdurchführung:} Siehe Skript.\\

\textsc{Beobachtung:}\hspace{5mm} 

\textsc{Auswertung:}\hspace{8mm} 


\newpage
\section{9.3 Konzentrationsabhangigkeit des ElektrodenPotentials}

Untersucht wurde hier, wie Änderungen der Konzentration, von der Salzlösung einer Halbzelle, die Spannung eines galvanischen Elements verändern.

\textsc{Versuchsdurchführung:} Siehe Skript.\\

\textsc{Beobachtung:}\hspace{5mm} Die Messergebnisse, sowie die Potentialdifferenz nach $\Delta U=C_n-C_{n-1}$ für Potentialdifferenz $=:\Delta U$, Konzentration $=:C_n$ und Vorherige Konzentration $=: C_{n-1}$, sind in folgender Tabelle aufgelistet:\\
\begin{center}
\textsc{Tabelle 1: Messergebnisse Versuch 9.3}\\
\vspace{5mm}
\begin{tabular}{ccc}
Konzentration [$mol\cdot L^{-1}$] & Spannung [$V$] & Potentialdifferenz [$V$]\\
 \hline
 1 : 1 & 0,0088 & --\\
 1 : 0,1 & 0,0025 & -0,0063\\
 1 : 0,01 & 0,0211 & 0,0186\\
 1 : 0,001 & 0,0338 & 0,0127\\
\end{tabular}
\end{center}

\textsc{Auswertung:}\hspace{8mm} Das gesamte System hat das Bestreben einen Konzentrationsausgleich zu vollziehen. So ist die Kupferlösung mit geringerer Konzentration bestrebt Kupferionen aufzunehmen, während die Lösung mit hoher Kupfersalzkonzentration bestrebt ist, Kupferionen als $Cu^0_{(s)}$ abzugeben. Somit lässt sich folgern das an der Elektrode der konzentrationsschwachen Lösung, Kupfer gemäß: $$Cu_{(s)}\rightleftharpoons Cu^+_{(aq)} + e^-$$ oxidiert wird. Während an der anderen Elektrode Kupferionen gemäß: $$Cu^+_{(aq)} + e^- \rightleftharpoons Cu_{(s)}$$ reduziert werden. Die Elektronen fließen also von der Halbzelle mit geringer Konzentration an Kupfersalz in Lösung zu der Halbzelle mit hoher Konzentration an Kupfersalz in Lösung. Oder anders beschrieben: von der Anode zur Kathode. Gleichzeitig werden die Anionen in der Salzbrücke von den gebildeten Kathionen der Anoden Halbzelle angezogen, da sonst in den Halbzellen Ladungen aufgebaut werden würden, die den Übergang von Elektronen zu der anderen Halbzelle verhindern würden. Dies ist analog auf die Kathionen der Salzbrücke und der Kathodenhalbzelle zu übertragen. Das Potenzial eines galvanischen Elements errechnet sich für gewöhnlich über: $$E=E_0+\frac{RT}{zF}\cdot \ln\frac{\{Ox\}}{\{Red\}}$$ Da wir hier die Abhängigkeit der Konzentration untersuchen und nicht die der Aktivität ersetzen wir Aktivität durch Konzentration, da diese näherungsweise identisch sind. Des Weiteren ist $E_0$ nach $E_0 = E_{Reduktion} - E_{Oxidation}= 0,8-0,8 = 0$ und fällt somit weg. Die Anzahl der übertragenen Elektronen ist in diesem Fall 1, damit das neutrale Element der variabel und fällt somit ebenfalls weg. Wir erhalten: $$E=\frac{RT}{F}\cdot \ln\frac{[Ox]}{[Red]}$$ Damit wir nun eine Zuordnung zwischen Verdünnungsgrad und dem theoretischen Potenzial wird nun die Oxidationskonzentration durch $10^{-n}$ und die Reduktionskonzentration durch $1$ ersetzt. Wir erhalten: $$E=\frac{RT}{F}\cdot \ln(10^{-n})$$ Nach dem Basiswechselgesetz gilt nun $\ln(x)=\frac{\lg(x)}{\lg(e)}\approx2,3026\cdot \lg(x)$ Durch Anwenden folgt folgende Gleichung: $$E=-2,3026n\cdot\frac{RT}{F}$$ das Potenzial verändern sich also pro Verdünnungsschritt um $2,303\frac{RT}{F}$. Bei den Oben genannten Messergebnissen (Tabelle 1) ist dies nicht der Fall. Mögliche Ursachen können sein: unscheinbare Fehler im Versuchsaufbau wie z.B. leichter Kabelbruch oder ein Fehlerhafter oder schlecht kalibrierter Voltmeter; Verunreinigungen der Halbzellenflüssigkeiten oder Verunreinigungen, bzw falsch angesetzte $KNO_3$ Lösung und Verunreinigungen an der Elektrodenoberfläche. Dies ist eine Entropie getriebene Reaktion, da in diesem System ein Bereich mit hoher Ordnung (Halbzelle mit hoher Salzkonzentration) und ein Bereich mit geringer Ordnung (Halbzelle mit geringer Salzkonzentration) zu erkennen ist. Das Bestreben aller Systeme, welche eine Temperatur $>0K$ haben ist, einen Zustand hoher Entropie zu erreichen. Dies äußert sich durch das Bestreben der Anode Kupferionen abzugeben und der Kathode Kupfer aufzunehmen, da dadurch die Konzentrationen angeglichen werden. Ein Ende der Reaktion ist dann erreicht, wenn die Konzentrationen ausgeglichen sind. Da hier die Entropie an ihrem Maximum ist. 

\newpage
\section{9.4 Batterien -- Das \textsc{LECLANT\`E} - Element}


\textsc{Versuchsdurchführung:} Siehe Skript.\\

\textsc{Beobachtung:}\hspace{5mm} Bei diesem Versuch wurden $3,68g MnO_2$ und $3,47g NH_4Cl$ verwendet. Anschließend wurde eine Spannung von 48 mV gemessen. 

\textsc{Auswertung:}\hspace{8mm} Um die zugrunde liegende Reaktion zu verstehen, kann die Reaktion grob in Anodenreaktion, Kathodenreaktion und Nebenreaktion unterteilt werden. An der Anode wird Zink nach der folgenden Reaktion oxidiert: $$Zn \rightleftharpoons Zn^{2+}_{(aq)} + 2e^-$$ An der Kathode hingegen wird Braunstein in zwei Schritten zu Wasser und Mangan(III)-Oxid reduziert. $$2MnO_{2(s)} + e^- + 2 H^+_{(aq)} \rightleftharpoons 2MnOOH \rightleftharpoons H_2O + Mn_2O_{3(s)} $$ Die Nebenreaktion beinhaltet die weiteren Reaktionsschritte des Zinkions sowie die Bildung der in der Reduktion gebrauchten Protonen: $$Zn^+_{(aq)} + 2 NH_4Cl_{2(s)} \rightleftharpoons [Zn(NH_3)_2]Cl_{2(s)} + 2H^+_{(aq)}$$ interessant ist nun, wie hoch die zu erwartende Batteriespannung ist. Die oben aufgeführte Formel: $$U= E_{Reduktion} - E_{Oxidation}$$ Muss noch erweitert werden, da die einzusetzenden Größen erst noch errechnet werden müssen. Diese sind über $$E_{Oxidation} = -0,763V+\frac{0,059}{2}V\cdot\log_{10}([Zn^{2+}])$$ $$E_{Reduktion} = 1,014V - 0,059\cdot pH$$ definiert (siehe Skript). Der ph-Wert ist mit 4,5 gegeben. Die Konzentration von $[Zn^{2+}]$ wie sie für die Berechnung von $E_{Oxidation}$ gefragt ist, errechnet sich aus dem Löslichkeitsprodukt von $Zn(OH)_2$ nach $$K_L=[Zn^{2+}]\cdot [OH^-]^2$$$$\left.10^{-17}=[Zn^{2+}]\cdot [OH^-]^2\right|\cdot [OH^-]^2$$$$[Zn^{2+}] = \frac{10^{-17}}{[OH^-]^2}$$ Die Konzentration von $[OH^-]$ lässt sich über die Gesetze des pH-Wertes berechnen.Da wir eine wässrige Lösung betrachten kann die Formel für den pH-Wert in die des pOH-Werts überführt werden: $$ \left. 14=pH+pOH\ \right| -pH $$ $$ \left. pOH=14-pH\ \right| pOH=-\log_{10}([OH^-]) $$ $$ \left. -\log_{10}([OH^-])= 14-pH\ \right|\cdot (-1) $$ $$ [OH^-] = 10^{(-14-pH)} = 10^{-9,5}$$ Dies geklärt, können wir nun die fehlenden Werte errechnen: $$[Zn^{2+}]=\frac{10^{-17}}{(10^{9,5})^2}=10^{-36}V$$ $$E_{Oxidation}=-0,763V+\frac{0,059}{2}\cdot\log(10^{-36})=-1,825V$$ $$ E_{Reduktion}= 1,014V-0,059\cdot 4,5=0,7485V$$ $$U=0,7485V-(-1,825V)=2,5735V$$ Dies ist die Leerlaufspannung, dieser Batterie. Die Elektronen werden an den Elektroden der Batterie gebildet. Diesem Effekt entgegen wirken die Größen: Konzentration der Lösung sowie der Diffusionsgradient bzw. die Temperatur. Bei der Reaktion wird Braunstein/Zink "Verbraucht" bzw $[Zn(NH_3)_2]Cl_2$ gebildet. Diese Stoffe müssen der Elektrode zugeführt/abgeführt werden, da wir keinerlei Umrührmechanismus in der Batterie verbaut haben, wird dies über die zwei oben genannten Faktoren bestimmt. Ist kaum Braunstein in der Lösung vorhanden, oder die Lösung ist sehr kalt, kann die Spannung zwischen den Elektronen nach "Verbrauch" dieser weniger schnell wieder aufgebaut werden. $$ Zn_{(s)} + 2MnO_{2(s)} \stackrel{H_2O}{\rightleftharpoons} ZnO_{(s)} + Mn_2O_{3(s)} $$Auch wenn nun die Reaktionen für die Bildung von $[Zn(NH_3)_2]Cl_2$ und $ZnO$ aufgestellt sind, kann jetzt keine zusammenfassende Gesamtgleichung aufgestellt werden, da nicht bekannt ist in welchem Verhältnis die Reaktionen Ablaufen. So können die stöchiometrischen Faktoren der Reaktanden nicht bestimmt werden. 
%X \underset{k_2}{\stackrel{k_1}{\rightleftharpoons}} Y
\end{document}