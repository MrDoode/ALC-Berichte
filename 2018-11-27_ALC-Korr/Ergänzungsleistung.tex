\documentclass[A4paper, 11p]{article}
\usepackage{fullpage}
\usepackage{todonotes}
\usepackage{chemfig}
\usepackage{mathtools}

\begin{document}

\title{Ergänzungsleistung\\ALC Tagesbericht-Berichtigung\\ vom 27.11.2018}
\author{Janosch Ehlers}
\date{Mo, 15.04.2019}
\maketitle

\begin{flushleft}
{\Large\textit{Aufgabe:}}
\end{flushleft}
Überarbeiten Sie einen oder nötigenfalls zwei Ihrer Laborberichte und erstellen Sie eine berichtigte Version. Welche Berichte zu bearbeiten sind, sagt Ihnen Ihr Praktikumsleiter. Diskutieren Sie in der berichtigten Version weiterhin, welche Fehler beim Erstellen des ursprünglichen Berichtes (wahrscheinlich) aufgetreten sind, und wie diese ggfs. in zukünftigen Laborberichten / Protokollen zu vermeiden sind. Stützen Sie sich dabei auch auf die Kommentare des Assistenten im Originalbericht. Die berichtigte Version ist zusammen mit dem Originalbericht abzugeben die Abgabefrist besprechen Sie bitte mit Ihrem Praktikumsleiter.\\

\begin{center}
\textsc{Inhalt}\\
\end{center}

\begin{enumerate}
\item 4.1 Einfache Redoxreaktionen
\item 4.2 Eloxieren von Aluminium
\item 4.3 Modell der Opferanode
\item 4.4 Wärmekissen und Kältemischungen
\item Fehlerdisskusion
\end{enumerate}

\newpage

\begin{center}
	\textsc{Redox- Reaktionen \& chemische Energetik}
\end{center}

\section{4.1 Einfache Redoxreaktionen}
Hier wurde eine Redoxreaktion, mit dem nachweiß für Stärke kombiniert. Im zweiten Teil wurde untersucht ob, Sulfitionen durch Kaliumpermanganat zu Sulfat oxidiert werden kann. Im dritten Teil hingegen wurden die verschiedenen Reaktionen von Salpetersäure und unedlen Metallen untersucht.
\begin{center} 
\textsc{Teil A} 
\end{center}

\textsc{Versuchsdurchführung:} Siehe Skript.\\

\textsc{Beobachtung:}\hspace{5mm} Nach Hinzugabe von Kaliumiodidlösung entstand eine farblose Lösung. Nach Hinzugabe von Chlorwasser färbte sich die Lösung dunkel-violett\\

\textsc{Auswertung:}\hspace{8mm} Die hier durchgeführte Reaktion ist eine Abwandlung der Nachweißreaktion von Stärke bzw. Iod. Bei dieser Reaktion wird elementares Iod gebildet, welches sich in die Helixstrukturen der Stärke einlagert. Die Redoxreaktion sieht wie folgt aus:\\

\begin{center} \begin{tabular}{lrcl}
Oxidation: & $2\cdot I^-_{(aq)}$ & $ \rightleftharpoons $ & $ I_2+2\cdot e^-$\\
Reduktion: & $Cl_2+2\cdot e^-$ & $\rightleftharpoons$ & $2\cdot Cl^-_{(aq)}$\\
\hline
 & Stärke $ + Cl_2\ +\ 2\cdot KI$ & $ \rightleftharpoons $ & $I_2+2\cdot KCl +$ Stärke\\
\end{tabular} \end{center}

\begin{center}
\textsc{Teil B}
\end{center}

\textsc{Versuchsdurchführung:} Siehe Skript.\\

\textsc{Beobachtung:}\hspace{5mm} Die angesäuerte Permanganatlösung ist stark violett gefärbt. Nach Hinzugabe von $NaSO_3$ klarte die Lösung sofort auf. Es bildete sich eine farblose Lösung und eine zweite unten stehende Phase, welche die gleiche Farbe hatte, wie die Ausgangslösung.\\

\textsc{Auswertung:}\hspace{8mm} Das hier ist eine simple Redoxreaktion mit Kaliumpermanganat als Oxidierungsmittel. $NaSO_3$ wurde hier in dieser Reaktion zu $NaSO_4$ oxidiert. Die Gesamtreaktion sieht wie folgt aus:\\ 

\begin{center}
\begin{tabular}{lrcl}
Oxidation:&$\displaystyle{5\cdot SO^{2-}_{3\ (aq)}\ +\ 5\cdot H_2O}$&$\rightleftharpoons$&$\displaystyle{5\cdot SO^{2-}_{4\ (aq)}\ +\ 10\cdot e^-\ +\ 10\cdot H^+}$\\
Reduktion:&$\displaystyle{2\cdot MnO^-_{4\ (aq)}\ +\ 10\cdot e^-\ +\ 16\cdot H^+}$&$\rightleftharpoons$&$\displaystyle{2\cdot Mn^{2+}_{(aq)}\ +\ 8\cdot H_2O}$\\
\hline
\end{tabular}
\begin{tabular}{rcl}
$\displaystyle{5\cdot SO^{2-}_{3\ (aq)}\ +\ 5\cdot H_2O\ +\ 2\cdot MnO^-_{4\ (aq)}\ +\ 16\cdot H^+}$&$\rightleftharpoons$&$\displaystyle{5\cdot SO^{2-}_{4\ (aq)}\ +\ 10\cdot H^+\ +\ 2\cdot Mn^{2+}_{(aq)}\ +\ 8\cdot H_2O}$\\
\hline
$\displaystyle{5\cdot SO^{2-}_{3\ (aq)}\ +\ 2\cdot MnO^-_{4\ (aq)}\ +\ 6\cdot H^+}$&$\rightleftharpoons$&$\displaystyle{5\cdot SO^{2-}_{4\ (aq)}\ +\ 2\cdot Mn^{2+}_{(aq)}\ +\ 5\cdot H_2O}$\\
\end{tabular}
\end{center}

\begin{center}
\textsc{Teil C}
\end{center}

\textsc{Versuchsdurchführung:} Siehe Skript.\\

\textsc{Beobachtung:}\hspace{5mm}
Im ersten Reagenzglas konnte eine leichte Schaumbildung beobachtet werden. Nach ein paar Minuten klarte die Lösung auf. Bei dem Zweite Reagenzglas, in welches die halb konzentrierte Salpetersäure hinzugegeben wurde, konnte eine starke Schaumbildung und die Bildung von rot-braunem Dampf beobachtet werden. Die konzentrierte Salpetersäure im dritten Reagenzglas hat derart stark reagiert, dass auch bei tropfenweiser Hinzugabe der Salpetersäure eine sehr starke Schaumbildung entstand. Auch diese Lösung klarte nach einer Zeit auf. Hier konnten danach auch keine Zink Reste mehr betrachtet werden.\\

\textsc{Auswertung:}\hspace{8mm} Die hier ablaufende Reaktion von, ist eine simple Redoxreaktion mit Salpetersäure als Oxidationsmittel. Die Reaktion im ersten Reagenzglas lässt sich wie folgt beschreiben:\\

\begin{center}
\begin{tabular}{crcl}
Oxidation: & $Zn$ & $\displaystyle{\rightleftharpoons}$ & $Zn^+ + 2e^-$\\
Reduktion: & $2\cdot H^+\ +\ 2\cdot e^-$ & $\rightleftharpoons$ & $H_2$\\
\hline
	& $Zn_{(s)}\ +\ 2\cdot HNO_3$ & $\rightleftharpoons$ & $Zn(NO_3)_2\ +\ H_{2}\uparrow_{(g)}$
\end{tabular}
\end{center}

Die Reaktionsgleichung, die der Reaktion im zweiten Reagenzglas zugrunde liegt, lässt sich wie folgt beschreiben:

\begin{center}
\begin{tabular}{crcl}
$3\cdot Oxidation$: & $Zn$ & $\rightleftharpoons$ & $Zn^+ + 2e^-$\\
$2\cdot Reduktion$: & $4\cdot HNO_3 + 3e^-$ & $\rightleftharpoons$ & $NO + 3\cdot (NO_3)^- + 2H_2O$\\
\hline
	& $3\cdot Zn + 8\cdot HNO_3$ & $\rightleftharpoons$ & $2\cdot NO + 3\cdot Zn(NO_3)_2 + 4\cdot H_2O$
\end{tabular}
\end{center}

Die dritte Reaktion lässt sich folgendermaßen beschreiben:

\begin{center}
\begin{tabular}{crcl}
$Oxidation$: & $Zn$ & $\rightleftharpoons$ & $Zn^+ + 2e^-$\\
$2\cdot Reduktion$: & $2\cdot HNO_3 + e^-$ & $\rightleftharpoons$ & $NO_2 + NO_3^- + H_2O$\\
\hline
	& $Zn + 4\cdot HNO_3$ & $\rightleftharpoons$ & $2\cdot NO_2 + Zn(NO_3)_2 + 2\cdot H_2O$
\end{tabular}
\end{center}

\newpage
\section{4.2 Eloxieren von Aluminium}

Hier wurde untersucht, wie die Stärke der $Al_2O_3$-Schicht und die Reaktivität des gleichen Aluminiumstücks zusammenhängen.

\begin{center}
\textsc{Teil A}
\end{center}

\textsc{Versuchsdurchführung:} Siehe Skript.\\

\textsc{Beobachtung:}\hspace{5mm} Die Aluminiumfolie begann sofort nach Hinzugeben zu der Kupferchlorid-Lösung, sich aufzulösen. Dabei entstand ein Gas und es bildete sich eine große Menge an rotem Niederschlag. Außerdem wurde viel Wärme erzeugt. Bei der Hinzugabe zu der Kupfersulfat-Lösung ist keine Reaktion zu erkennen.\\

\textsc{Auswertung:}\hspace{8mm} Kupfer ist edler als Aluminium, da Kupfer ein elektrochemisches Potenzial von +0,16V bzw. +0,34V, relativ zu Standardwasserstoff Elektrode, hat. Demnach hat Kupfer ein höheres Reduktionsbestreben als Aluminium, mit einem Redoxpotenzial von -1,68V relativ zur Standardwasserstoff Elektrode, und ist somit edler als Aluminium.\footnote{siehe Skript, Anhang D: Spannungsreihe, Seite 121}. Die $Al_2O_3$-Schicht auf dem Aluminium kann allerdings nur von den Chlorid Ionen durchbrochen werden. Somit ist diese Reaktion nur bei Präsenz von Chlorid Ionen oder anderen Reaktanden, welche die $Al_2O_3$-Schicht durchbrechen können, möglich. Dies ist bei der Kupfersulfat Lösung nicht der Fall. Demnach ist auch keine Reaktion zwischen Aluminium und Kupfersulfat nicht möglich. Das Gas, welches beobachtet wurde, ist wahrscheinlich Wasserdampf, der sich unter der Hitzeentwicklung gebildet hat. Die gesamte Redoxgleichung ist die folgende:\\

\begin{center}
\begin{tabular}{lrcl}
Oxidation:&$2\cdot Al_{(s)}$&$\rightleftharpoons$&$2\cdot Al^{3+}_{(aq)}\ +\ 6\cdot e^-$\\
Reduktion:&$3\cdot Cu^{2+}_{(aq)}\ +\ 6\cdot e^-$&$\rightleftharpoons$&$3\cdot Cu_{(s)}$\\
\hline
 &$2\cdot Al_{(s)}\ +\ 3\cdot Cu^{2+}_{(aq)}$&$\rightleftharpoons$&$2\cdot Al^{3+}_{(aq)}\ +\ 3\cdot Cu_{(s)}$\\
\end{tabular}
\end{center}

\begin{center}
\textsc{Teil B}
\end{center}

\textsc{Versuchsdurchführung:} Siehe Skript.\\

\textsc{Beobachtung:}\hspace{5mm} \textbf{i)}Nach Starten des Versuchs bildet sich eine braune Schicht auf der Aluminiumplatte. \textbf{ii)} Bei einer angesetzten Spannung von 6,0V und 0,73A konnte eine leichte Gasentwicklung an der Kathode festgestellt werden. \textbf{iii)} Hier konnte nun eine deutlich größere Gasentwicklung festgestellt werden. Bei der Kathode entstand deutlich mehr Gas als bei der Anode. Des Weiteren färbte sich die Lösung leicht Grau und wurde warm. Auf der Aluminiumplatte entstand außerdem eine Schwarze Ablagerung. \textbf{iv)} Die entstandene Ablagerung auf der Aluminiumplatte von iii) löste sich auf und es bildete sich eine kupferfarbene Schicht auf der Aluminiumplatte, allerdings deutlich weniger als bei i).\textbf{v)} nach Kratzen der Aluminiumplatte in der Lösung konnte sofort die starke Bildung einer kupferfarbenen Schicht entlang der Kratzspuren entdeckt werden. 

\textsc{Auswertung:}\hspace{8mm} In Teilversuch \textbf{i)} wurde das Kupfer nach folgender Reaktionsgleichung reduziert:

\begin{center}
\begin{tabular}{lrcl}
Oxidation:&$2\cdot Al_{(s)}$&$\rightleftharpoons$&$2\cdot Al^{3+}_{(aq)}\ +\ 6\cdot e^-$\\
Reduktion:&$3\cdot Cu^{2+}_{(aq)}\ +\ 6\cdot e^-$&$\rightleftharpoons$&$3\cdot Cu_{(s)}$\\
\hline
 &$2\cdot Al_{(s)}\ +\ 3\cdot Cu^{2+}_{(aq)}$&$\rightleftharpoons$&$2\cdot Al^{3+}_{(aq)}\ +\ 3\cdot Cu_{(s)}$\\
\end{tabular}
\end{center}
Die Reaktion läuft freiwillig ab, da hierbei das elektrochemische Potenzial verringert wird. In Teilversuch \textbf{iii)} wird nun umgepolt und die Elektrolyse Spannung wird erhöht. Dadurch wird die $Al_2O_3$-Schicht verstärkt. Dabei wird das Wasser durch die Elektrolyse in Wasserstoff und Sauerstoff gespalten. Das Wasserstoff steigt nun als $H_2$ an der Kathode als Gas auf und der Sauerstoff wird in Form von Ionen zur Anode transportiert. Dort reagieren die Anionen mit dem Aluminium und formen die $Al_2O_3$-Schicht. Zuvor wird mit umgekehrter Polung elektrolysiert um die vorhandene $Al_2O_3$-Schicht abzutragen. Das Wiederumkehren der Spannung in \textbf{iv)} verursachte genau den gleichen Effekt: Die Passivierungsschicht des Aluminiums wurde abgetragen und auf dem Aluminium bildet sich eine Kupfer Beschichtung. \footnote{https://www.precifast.de/eloxieren-funktionsweise-vorteile/}

\newpage
\section{4.3 Modell der Opferanode}

Hier wurde untersucht, wie sich die Reaktivität von unedlen metallen verändert, wenn ein Stoff mit geringerem elektrochemischem Energiepotenzial (gemessen an der Standardwasserstoffelektrode) vorhanden ist.
 
\textsc{Versuchsdurchführung:} Siehe Skript.\\

\textsc{Beobachtung:}\hspace{5mm} Nach Eintauchen des Anspitzers in die Lösung konnte nach einigen Minuten die Bildung von kleinen rostroten Punkten auf den Klingen des Anspitzers und kleine Weiße punkte auf dem Körper des selbigen, entdeckt werden. Die PH-Wert-Messung mit dm Universalindikatorpapier ergab einen Wert von 6.

\textsc{Auswertung:}\hspace{8mm} Ein Lokalelement ist eine Galvanische Zellen, dessen Elektroden elektrisch ohne Widerstand verbunden worden sind. In unserem Versuch lag dieses Verhältnis vor, sobald der Anspitzer in die Elektrolytlösung gelegt wurde. Hier findet eine Oxidation des Magnesiums statt und eine Reduktion an dem Stahl. Beide Metalle haben allerdings keinen elektrischen Widerstand Zwischengeschalten. Als folge löst sich das Magnesium unter Bildung von Wasserstoff an der Stahlanode auf\footnote{http://www.chemie.de/lexikon/Lokalelement.html}. Das Magnesium übernimmt somit hier die Rolle der sog. Opferanode. Das Magnesium hat ein höheres Oxidationsbestreben, als der Stahl, da dies unedler ist. Dadurch wird es zuerst oxidiert. Der Stahl, der unter anderem Umständen Rosten (oxidieren) würde übernimmt somit die Rolle der Anode einer kurzgeschlossenen galvanischen Zelle. An dem Magnesiumblock findet nun folgende Oxidation Statt: $$Mg_{(s)}\ +\ H_2O\ \rightleftharpoons\ MgO\ +\ 2\cdot h^+\ +\ 2\cdot e^-$$ Während an der Stahl-Klinge folgende Reduktion abläuft: $$2\cdot H^+\ +\ 2\cdot e^- \rightleftharpoons H_{2}\uparrow_{(g)}$$ Zusammen sieht die Gesamtreaktion wie folgt aus: $$Mg_{(s)}\ +\ H_2O \rightleftharpoons MgO_{(s)} \ +\ H_{2}\uparrow_{(g)}$$ Die Lösung fängt darauf hin an leicht sauer zu werden, da bei der Reaktion an dem Magnesium Block Protonen entstehen. Da der Magnesium block nun eine größere Oberfläche hat als die Klinge, kann diese Reaktion leicht bevorzugt werden. So lässt sich die PH-Wert-Messung von PH=6 erklären. Der Niederschlag, bzw. die Ablagerung auf dem Magnesium Körper ist somit auf die Bildung von $MgO$ zurückzuführen.

\newpage
\section{4.4 Wärmekissen und Kältemischungen}

Hier wurde untersucht, wie sich Lösungen erstellen lassen, die bestimmte wärmende bzw. kühlende Wirkungen haben.

\begin{center}
\textsc{Teil A}
\end{center}

\textsc{Versuchsdurchführung:} Siehe Skript.\\

\textsc{Beobachtung:}\hspace{5mm} Nachdem wir die Lösung, nach Erreichen der Raumtemperatur, haben kristallisieren lassen, konnten wir eine Temperatur von $41^\circ C$ messen.\\

\textsc{Auswertung:}\hspace{8mm} $$CH_3COONa\cdot 3H_2O \underset{-\text{Wärme}}{\stackrel{+\text{Wärme}}{\rightleftharpoons}} CH_3COO^-_{(aq)}\ +\ nH_2O$$

Wenn Wärmekissen aus kristallisieren dann gibt das Salz in der Lösung, im Wärmekissen, die Energie frei die das Salz in Lösung hält. Erhitzt man das Wärmekissen wieder, wird dem System wieder Energie hinzugefügt und das Salz löst sich wieder in der Flüssigkeit. Die Flüssigkeit verbleibt, während das Salz kristallin vorliegt als (z.B.) Kristallwasser vor. Nach dem Lösen liegt eine übersättigte Lösung vor. Hier ist mehr Salz in einer Flüssigkeit gelöst, als die Flüssigkeit binden kann. Um in diesen Zustand zu kommen, muss viel Energie aufgebracht werden. Eine übersättigte Lösung ist sehr instabil, da das Salz das bestreben, hat zu kristallisieren und seine Energie abzugeben. Hergestellt werden diese Lösungen, indem sehr viel Salz unter zufuhr, von Wärme wenig Flüssigkeit zugeführt wird und danach auf Raumtemperatur abgekühlt wird.\\

\begin{center}
\textsc{Teil B}
\end{center}

\textsc{Versuchsdurchführung:} Hier wurde in Versuch B2) $CaCl_2\ \cdot\ 2H_2O$ verwendet anstelle von $CaCl_2\ \cdot\ 6H_2O$. In allen anderen Punkten siehe Skript.\\

\textsc{Beobachtung:}\hspace{5mm} Nach Erstellen der 1. Lösung (Eis und $NaCl$) konnte eine Temperatur von $-11^\circ C$ gemessen werden. Die zweite Lösung (Eis + $CaCl$) erreichte nach Ansetzen eine Temperatur von $13^\circ C$. Die dritte Lösung (Flüssiger Stickstoff + Ethanol) erreichte eine Temperatur von $-39,7^\circ C$. Die Raumtemperatur betrug $19^\circ C$.\\

\textsc{Auswertung:}\hspace{8mm} In der folgenden Tabelle sind einige Kältemischungen mit Literatur Wert nach:\footnote{http://www.chemie.de/lexikon/K\%C3\%A4ltemischung.html} und \footnote{https://www.sigmaaldrich.com/content/dam/sigma-aldrich/docs/Aldrich/Method/1/aldrich-running-\%20low-temperature-reaction.pdf} aufgetragen.\\

\begin{tabular}{cc}
100g Eis + 33g $NaCl$ & $-21,3^\circ C$\\
100g Eis + 143,9g $CaCl\cdot 6H_2O$ & $-55^\circ C$\\
Ethanol + $N_{2(l)}$ & $-116^\circ C$ \\
100g Eis + 28,2g $MgCl_2$ & $-33^\circ C$\\
Aceton + $CO_{2(s)}$ & $-78^\circ C$\\
Ether + $CO_{2(s)}$ & $-82^\circ C$\\
\end{tabular}\\

Wie deutlich zu sehen ist, sind alle drei Kältelösungen weit von Ihren Literatur werten entfernt. Dies kann viele Gründe haben: Zum einen können die Chemikalien verunreinigt oder die Reaktionsgefäße unsauber sein. Allerdings ist auch nicht auszuschließen, dass beim Ansetzen dieser Lösungen die Fehler passiert sein können. Auch muss berücksichtigt werden, dass bei Lösung Zwei ($CaCl$) nicht die im Skript beschriebene Chemikalie genutzt wurde, dies kann auch Auswirkung auf die Temperatur haben. Bei einem Kühlprozess wird Energie inform von Wärme ähnlich, wie bei einem Druckausgleich oder Diffusion von Ionen vom Punkt der höchsten Konzentration zum Punkt der niedrigsten Konzentration transportiert. Bei Kühlung mit Eis wird das Eis erwärmt und entzieht dabei der Umgebung Wärme, welche selbst somit zum Kühlmittel wird und somit ihrer Umgebung Wärme entzieht. Wärme selbst ist die Bewegungsgeschwindigkeit der Teilchen eines Stoffes.

\newpage
\section{Fehlerdiskussion}
Die Fehler, die mir beim Erstellen des ersten Laborberichtes Wiederfahren sind, lassen sich größtenteils auf: nicht Bearbeiten der Aufgaben und fehlender Bearbeitungszeit Herunterbrechen. Es ist nicht auszuschließen das Prokrastination meinerseits auch Ursache der vielen Fehler sein kann. Der Abgabe Termin dieses Berichtes fiel zusammen mit zwei weiteren Abgaben mit ähnlichem Arbeitsaufwand und persönlichen Ereignissen. Viele Fehler sind daher definitiv nicht auf Nicht-Wissen/Falschinformationen meinerseits zurück zuführen, sondern schlicht weg auf Nicht-Bearbeiten zurückzuführen (wie es z.B. bei den Auswertungen von 4.2 oder 4.3 der Fall war.)Ein weiteres Indiz dafür ist die hohe Rechtschreibfehler dichte, oder das Unvermögen meinerseits nach Seite 1, die Seite Zwei zu heften (unabhängig davon, ob diese Fehler gewertet wurden oder nicht). Eine weitere Fehlerquelle war natürlich auch das Unwissen über Konventionen, welche ich erst nach Analysieren der folgenden Praktikumsberichtkorrekturen erworben haben, wie z.B. Dass bei der Aufstellung einer Redoxgleichung die unbeteiligten Ionen eines Salzes nicht mitgeschrieben werden, oder das Angeben des Aggregatzustandes eines Teilchens in der Reaktionsgleichung bzw. des Solvatisierungszustands.Auch muss gesagt werden, dass natürlich auch einfache Denkfehler zu Fehlern führen können. Wie dem Originalbericht außerdem entnommen werden kann, wurde mit meiner Gruppe nach Rückgabe des Berichtes eine Rücksprache gehalten. Der Assistent, der mit uns dieses Gespräch gehalten hat, hat mir versichert, hätte ich mich ihm vor Abgabe des Berichtes anvertraut und ihm meine Situation geschildert, wäre es bestimmt möglich gewesen die Frist zu verlängern, sodass evtl. trotzdem einen passablen Bericht hätte abgeben können. Somit ist natürlich auch das "Nicht-Bescheid-Geben" als Fehler dieses Berichts anzusehen. Für folgende Abgaben kann ich also beachten:
\begin{enumerate}
\item Bessere Eigenorganisation
\item Mehr Zeit für die Bearbeitung einplanen
\item Aufgaben Zuende lesen
\item Absprachen treffen
\end{enumerate}
\end{document}
