\documentclass[11pt, a4paper]{article}
\usepackage{fullpage}

\begin{document}

\title{ALC Tagebericht\\vom 13.11.18}
\author{Jacqueline Preis, Janosch Ehlers}
\maketitle

	\begin{center}
	\textsc{Konzentrationsangaben und Verdünnungen}
	\end{center}

\section*{Station 2.0\\Umgang mit Gasflaschen}

Hier wurden wir mit der Funktionsweise, den Sicherheitsvorkehrungen, den Gefahren und den Umgang mit Gasflaschen vertraut gemacht.\\

Generell kann gesagt werden das Gasflashen einen Innendruck von ~200Bar haben, wenn sie geliefert werden. Dies ist gleichzeitig auch der Maximaldruck den Gasflaschen sicher halten können. In Laboren werden generell sehr viele verschiedene Gase benötigt. Am häufigsten allerdings wird Sauerstoff, Wasserstoff und Stickstoff verwendet. Propan wird oft als Brenngas genutzt um Bunsenbrenner zu betreiben. Um Gasflaschen nutzen zu können, wird der Flaschendruck auf einen Arbeitsdruck, welcher bedeutend kleiner ist als der Flaschendruck, heruntergeregelt. Dazu benutzt man einen Druckminderer. a dieser direkt an die Flasche angebracht wird, muss dieser zu der Gasflasche passen. Da brennbare Gase idR. ein links laufendes Gewinde haben, wehrend alle anderen Gassorten ein rechtslaufendes Gewinde besitzen, kann nicht jeder Druckminderer für jede Gasflasche benutzt werden. Gasflaschen können grob mit folgenden Farben kategorisiert werden\footnote{www.linde-gas.de/de/safety\_and\_quality/health\_and\_safety/cylinder\_safety/identifying\_cylinders/index.html [abgerufen am 19.11.2018 19:45]}:
\begin{enumerate}
\item Gelb: Ätzende und giftige Gase
\item Rot : Entzündliche Gase (Bsp. $H_2$)
\item Hellblau: Oxidierende Gase
\item Hellgrün: Inertgase
\item Weiß : Medizinische- / Inhalationsgase
\end{enumerate}
\newpage


\section*{Station 2.1\\Verdünnungsreihe}

Zu prüfen war die Salinität verschiedener Meerwasserlösungen durch Messen der Leitfähigkeit.
Erwartet worden ist, dass die Salinität proportional zum Verdünnungsfaktor abnimmt. In zwei Versuchen wurde eine sequenzielle Verdünnungsreihe und eine direkte Verdünnungsreihe hergestellt.
	\begin{center}
	\textsc{Sequenzielle Verdünnungsreihe}\\
		\begin{tabular}{cccc}
		Vorlösung(g) & Dest. Wasser(g) & Leitfähigkeit($\mu A$) & Verdünnungsfaktor\\
		\hline
		50,35g Meerwasser & -	& 233,0		&  - \\
		24,13 & 25,10	& 233,5		& 2\\
		25,00 & 25,20	& 233,3		& 4\\
		25,59 & 25,41	& 233,3		& 8\\
		25,03 & 25,00	& 232,3		& 16\\
		25,03 & 25,97	& 229,0		& 32\\
		25,01 & 25,99	& 229,9		& 64\\
		25,30 & 25,67	& 219,9		& 128\\
		25,31 & 25,87	& 206,7		& 256\\
		25,80 & 25,21	& 184,8		& 512\\
		25,00 & 25,05	& 133,0		& 1024\\
		25,74 & 25,32	& 0,3		& 2048\\
		25,28 & 25,99	& 0,2		& 4096\\
		\end{tabular}
	\end{center}


	
	\begin{center}
	\textsc{Direkte Verdünnungsreihe}\\
		\begin{tabular}{cccc}
		Meerwasser(g) & Dest. Wasser(g) & Leitfähigkeit($\mu A$) & Verdünnungsfaktor\\
		\hline
		49,41g Meerwasser & -		& 234,0 & -\\
		24,91	& 25,77		& 233,0		& 1\\
		25,14	& 50,30		& 233,1		& 2\\
		25,94	& 75,19		& 232,8		& 3\\
		10,82	& 50,39		& 233,0		& 5\\
		10,74	& 95,80		& 232,5		& 10\\
		5,40	& 95,18		& 232,0		& 20\\
		2,48	& 100,07	& 229,4		& 50\\
		1,09	& 100,03	& 225,2		& 100\\
		0,5		& 100,68	& 220,0		& 200\\
		0,11	& 100,65	& 196,0		& 1000\\
		\end{tabular}
	\end{center}
	
\textsc{Auswertung:}\hspace{8mm} 
Der Verdünnungsfaktor ist der Wert des Anteils von Lösungsmittel in einer Lösung. Somit ist der Verdünnungsfaktor: $$\displaystyle{f_{verd} = \frac{Lg_{mittel}}{Lg_{Stoff}}}$$ Da die Salinität das Vorhandensein von Ionen in einer Wasser Lösung darstellt, sollte die Salinität proportional zu $\frac{1}{f_{verd}}$ sein. 


\section*{Station 2.2\\Löslichkeit von Gasen}

Untersucht wurde, ob Gase sich ähnlich lösen lassen wie Feststoffe. In zwei Versuchen wurde das Lösen von Gas über Druck und die Sättigung von Lösungen untersucht.

	\begin{center}
	Teil 1
	\end{center}

\textsc{Versuchsdurchführung:} Siehe Skript.\\

\textsc{Beobachtung:}\hspace{5mm} Nach Komprimieren des Gases konnten Volumenveränderungen festgestellt werden. So betrug das Gesamtvolumen, statt den 100 ml Startvolumen, nun 97 ml. Das Wasser, welches vorher ein Raum von 16ml einnahm, nahm jetzt einen Raum von 17 ml ein.\\
\textsc{Auswertung:}\hspace{8mm} Die 4ml Gas haben sich natürlich wie hier beschrieben: $$O_{2(g)} + N_{2(g)} + R_{(g)} + H_2O_{(l)} \rightleftharpoons H_2O_{(l)} + O_{2(aq)} + N_{2(aq)} + R_{(aq)}$$ im wasser gelöst ("R" steht in diesem Fall für Rest, gemeint sind damit Spurenelemente und andere Gase in der Luft). Dort nimmt das Gas viel weniger Platz ein. Damit wäre auch die Volumen Zunahme des Wassers erklärt. Allerdings ist unklar, ob die gesamten 4ml Gas sich in dem Wasser gelöst haben. Es kann auch sein, dass der Innendruck im Kolbenprober nicht gereicht hat um den Kolben, des Probers, weiter herauszudrücken und das Gas eigentlich ein größeres Volumen einnehmen würde. Vielleicht war der Kolben an dem Punkt so langsam, dass hier die Haftreibung gegriffen hat und somit einfach stehen blieb. Auch können Teile des Gases durch das Ventil bzw. dem Kolben entweichen. Es kann also auf dieser Ebene schwer festgestellt werden, was genau mit dem gesamten Gas passiert ist, allerdings kann gesagt werden, dass wahrscheinlich das meiste an Gas in Lösung mit dem Wasser gegangen ist. 

	\begin{center}
	Teil 2
	\end{center}
	
\textsc{Versuchsdurchführung:} Siehe Skript.\\

\textsc{Beobachtung:}\hspace{5mm} Die erste Tablette hat ein Gasvolumen von 10ml produziert. Die zweite Tablette hingegen hat ein Gasvolumen von 25ml produziert.\\

\textsc{Auswertung:}\hspace{8mm} Die Tablette bestand unter anderem aus :

\begin{enumerate}
\item Zitronensäure (2-Hydroxypropan-1,2,3-tricarbonsäure)
\item Magnesiumcarbonat ($Mg^+CO^{2-}_3$)
\item Natrium Hydrogencarbonat ($Na^+CO_3^-$)
\item Apfelsäure (2-Hydroxybutandisäure)
\end{enumerate}

Die Ergebnisse aus Teil 2 wirken zunächst unerwartet. Erwartet wurde, dass beide Tabletten gleich viel Gas produzieren.
Allerdings wird die Lösung beim Reagieren mit der Brausetablette mit Gas gesättigt. Bei der zweiten Tablette wird nun kein Gas mehr von der Lösung aufgefangen und steigt nun nach oben.
Treibende Stoffe der Reaktion sind hier das $MgCO_3$ und das $NaHCO_3$ sowie die Zitronensäure und die Apfelsäure. Das $NaHCO_3$ reagiert zerfällt in Wasser zu Wasser ,$Na_2CO_3$ und $CO_2$ Das $CO_2$ fällt als Gas aus oder geht in Lösung. in einem Sauren Milieu, welches durch die Zitronen Säure und die Apfelsäure vorliegt, findet nun eine klassische Säure-Base Reaktion statt. Es wird $H_2O$, in diesem Fall $CO_2$ und das Salz aus Baserest und Säurerest gebildet. Die Reaktionsgleichungen sehen wie folgt aus\footnote{www.seilnacht.com/Chemie/ch\_nahco.htm}\footnote{http://www.seilnacht.com/Chemie/ch\_mgco3.htm}:\\
\begin{center}
$NaHCO_3 \rightleftharpoons Na_2CO_3 + CO_2 +  H_2O$ \\
$MgCO_3 +$\textit{Zitronensäure}$ \rightleftharpoons Magnesiumhydrogencitrat + H_2O + CO_2$\\
$Na_2CO_3 +$\textit{Zitronensäure}$ \rightleftharpoons Dinatriumhydrogencitrat + H_2O + CO_2$\\
$MgCO_3 +$\textit{Apfelsäure}$ \rightleftharpoons Magnesium-2-hydroxybutandiat + H_2O + CO_2$\\
$Na_2CO_3 +$\textit{Apfelsäure}$ \rightleftharpoons Dinatrium-2-hydroxybutandiat + H_2O + CO_2$\\

\end{center}

\section*{Station 2.3\\Eudiometerversuch}

Hier wurde versucht die Summenformel von Wasser experimentell, über eine kontrollierte Form des Knallgas Expermiments, nachzuweisen.



\textsc{Versuchsdurchführung:} Siehe Skript.\\

\textsc{Beobachtung:}\hspace{5mm} Nach Zünden des Gasgemischs konnten folgende Volumina abgelesen werden:\\
	\begin{center}		
	\begin{tabular}{c|ccc|cc}
	SV & $H_2$ & : & $O_2$ & RV & EEV\\ 
	\hline
	4 VE & 3 & : & 1 & 1 VE & 1 VE \\
	2 VE & 1 & : & 1 & 1 VE & 0,5 VE \\
	3 VE & 1 & : & 2 & 2,5 VE & 1,5 VE \\
	3 VE & 2 & : & 1 & 1 VE & 0 VE\\ 
	\end{tabular}
	\end{center}
Folgende Begriffe wurden hier Abgekürtzt:\\
SV = "Start Volumen" RV = "Rest Volumen" EEV = "Erwartungs Endvolumen" VE = "Volumen Einheit"

\textsc{Auswertung:}\hspace{8mm} Das EEV ist logischerweise die Differenz zwischen dem gesamt Volumen und dem größtmöglichen Volumen, welches eine ideale Menge an Wasserstoff und Sauerstoff enthält ($2H_2$:$1O_2$). Das einzige Ergebnis dieses Versuchs, der dem Erwartungswert entspricht, ist das Ergebnis von $3H_2$ : $1O_2$ hier konnte 1 VE Restvolumen gemessen werden. Bei den anderen Versuchen konnten Abweichungen zu dem Erwartungswert bis zu 1 VE gemessen werden. Dies kann zum einen an Unreinheiten in den Gasen liegen, da sowohl der Schlauch, der an dem Druckminderer zum Befüllen der Gasprober befestigt war, als auch der Schlauch an den kolbenprobern nicht mit reinen Gas ausgespült worden ist.So können viele Gase in den Kolbenprober gelangen, welche nicht Teil der Reaktion sind. Dies Sollte der Hauptgrund sein, warum wir zu solchen Messergebnissen gekommen sind. Andere Gründe mit geringerer Schwere könnten sein: Messungenauigkeiten beim Abmessen der Gasvolumina, Ausfallen der Gase aus der Flüssigkeit, ungenügende Mischung des Gasgemisches vor der Zündung. Allerdings können wenigstens die letzten beiden Fehlerquellen vernachlässigt werden. Schlussendlich ist zu sagen, dass es erstaunlich ist, wie stark sich ein Fehler fortsetzt, da mit diesen Messergebnissen ein Verhältniss von $3H_2$ : $O_2$ nicht undenkbar wäre, würden wir tatsächlich versuchen das Verhältnis von Wasserstoff und Sauerstoff in Wasser empirisch zu belegen.

\section*{Station 2.4\\Diffusion}

Zu untersuchen war im ersten Teil, die Art und Weise wie sich Feststoffe nach lösen in einer Flüssigkeit in dieser Verteilen.
Im zweiten Teil wurde die Verteilung von Gasen in einer Gasphase untersucht.

	\begin{center}
	Teil 1
	\end{center}


\textsc{Versuchsdurchführung:} Siehe Skript.\\

\textsc{Beobachtung:}\hspace{5mm} Bei Hinzugeben der drei Salze, konnte zu erst nur in der Flüssigkeit um das $FeCl_3$ ein gelb-brauner Schleier beobachtet werden, welcher auszubreiten schien. Die Flüssigkeit um die anderen Salze hat sich nicht sichtbar verändert. Nach wenigen Minuten konnte an der gedanklichen Mitte zwischen $FeCl_3$ und $FSCN$ die Bildung einer Rot-braunen Front erkannt werden. Diese vergrößerte sich zunächst in Richtung $FeCl_3$. Nach etwa 5 Minuten konnte die Bildung eines blauen Schleiers zwischen dem $FeCl_3$ und dem $K_4[Fe(CN)_6]$ erkannt werden. Mit der Zeit nahm die Verfärbung des Wassers zu, allerdings hat sich als Letztes der Teil der Petrischale verfärbt in den das $K_4[Fe(CN)_6]$ geschüttet wurde. Alle Verfärbungen des Wassers waren nur am Boden der Petrischale präsent.\\

\textsc{Auswertung:}\hspace{8mm} Das $FeCl_3$ bei hinzugeben von Wasser eine gelb-braune Farbe erzeugt ist wenig verwunderlich, da $FeCl_3$ selbst diese Farbe hat und in Lösung anscheinend seine Farbe beibehält. Dass die erzeugten Farbschleier ausschließlich am Boden der Petrischale zu sehen sind, liegt an der Tatsache, dass die gebildeten Farbkomplexe dichter sind als Wasser und damit zu Boden sinken. Bei der rot-braunen Front wie sieht weiter oben beschrieben wurde handelt es sich um ein Farbkomplex, welcher durch eine Reaktion von $KSCN$ und $FeCl_3$ gebildet wird. Die volle Reaktionsgleichung sieht wie folgt aus: 
$$Fe(III)Cl_3\ +\ KSCN \rightleftharpoons Cl_2\left[Fe(III)(SCN)\right]\ +\ KCl$$
Bei dem Blauen-Schleier wie er weiter oben beschrieben wurde handelt es sich ebenfalls um diffundierte Salze, die mit einander reagiert haben. In diesem Fall um $FeCl_3$ und $K_4[Fe(CN)_6]$ hierbei bildet sich der bekannte Farbstoff Berliner Blau. Die gesamte Reaktion sieht wie folgt aus\footnote{http://www.chemtube3d.com/solidstate/SS-PruBlu.htm}: $$4Fe(III)Cl_3\ +\ 3K_4\left[Fe(II)(CN)_6\right]\ \rightleftharpoons\ Fe(III)_4\left[Fe(II)(CN)_6\right]_3\ +\ 12KCl$$ 


	\begin{center}
	Teil 2
	\end{center}


\textsc{Versuchsdurchführung:} Siehe Skript.\\

\textsc{Beobachtung:}\hspace{5mm} Nach Aneinanderlegen der Reagenzgläser konnte keine Reaktion festgestellt werden. Erst nach Abbruch des Experiments und dem Entfernen der Reagenzgläser konnte die Bildung von Dampf an dem Reagenzglas mit Salzsäure erkannt werden.\\

\textsc{Auswertung:}\hspace{8mm} Der Weiße Rauch ist eine Reaktion von $HCl_{(g)}$ und $NH_{3 (g)}$. Die Gesamtreaktion sieht wie folgt aus: $$HCl_{(g)}\ +\ NH_{3 (g)}\ \rightleftharpoons\ ClNH_{4 (s)}$$ Der Dampf sollte sich unter normalen Umständen bei dem Ammoniak bilden, da $HCl$ einen größeren Diffusionskoeffizienten hat. Da dieser eine höhere Konzentration hat und damit ein höheres Bestreben zu diffundieren.

\end{document}
