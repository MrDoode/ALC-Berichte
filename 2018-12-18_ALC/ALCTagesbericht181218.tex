\documentclass[11pt, a4paper]{article}
\usepackage{fullpage}

\begin{document}

\title{ALC Tagesbericht\\ vom 27.11.2018}
\author{Janosch Ehlers, Jacqueline Preis}
\maketitle

	\begin{center}
	\textsc{Löslichkeitsprodukt, Säuren und Basen}
	\end{center}

\section{7.1 Löslichkeitsprodukt von Silberhalogeniden}

In diesem Versuch wurde durch Mischen von Silbernitrat-Lösung und verdünnter Salzsäure Silberchlorid ausgefällt. Anschließend, um weitere Silber-Ionen zu fällen, wurde Kaliumiodid-Lösung hinzugegeben. 

\textsc{Versuchsdurchführung:} Siehe Skript.\\

\textsc{Beobachtung:}\hspace{5mm} Durch Zugabe von verdünnter Salzsäure zu Silbernitrat-Lösung kommt es zu einer weißen Färbung. Nachdem anschließend filtriert wurde und erneut Salzsäure hinzugefügt wurde, fiel kein Silberchlorid mehr aus. Bei der darauffolgenden Zugabe von Kaliumiodid-Lösung wurde die Flüssigkeit grünlich und trüb.\\

\textsc{Auswertung:}\hspace{8mm} Gesucht ist die Konzentration an $Cl^-$-Ionen an dem $AgCl$ als Salz ausfällt. Das Löslichkeitsprodukt von $AgCl$ nach $Ag^+_{(aq)} + Cl^-_{(aq)} \rightleftharpoons AgCl$ ergibt sich aus den Konzentrationen der Edukte der Reaktion. Da das Löslichkeitsprodukt von $AgCl$  bei Standardbedingungen bei $1,6\cdot 10^{-10}\frac{Mol^2}{L^2}$ liegt und die Konzentration von Silber gegeben ist, gilt: $$0,1\frac{Mol}{L}\cdot c(Cl^-) = 1,6\cdot 10^{-10}\frac{Mol^2}{L^2}$$ Da nun der Punkt gesucht wird, an dem das Silber als Silberchlorid ausfällt und somit das Produkt der beiden Konzentrationen großer als $1,6\cdot 10^{-10}\frac{Mol^2}{L^2}$ ist, wird die Gleichung nach $c(Cl^-)$ aufgelöst: $$c(Cl^-)=\frac{1,6\cdot 10^{-10}\frac{Mol^2}{L^2}}{0,1\frac{Mol}{L}}=1,6\cdot 10^{-9}\frac{mol}{L}$$\\
Da die Konzentration von Silberionen durch die Fällungsreaktion verringert wird, wir aber dennoch durch die Stetige hinzugebe von Salzsäure die Konzentration von Chlorid-Ionen mehr oder weniger konstant halten wird so lange $AgCl$ ausfallen, bis wieder eine gesättigte Lösung entsteht. Somit ist die Konzentration von Chlorid in der Lösung Klar bestimmbar. Über $$K_L\ =\ c(Cl^-)\cdot c(Ag^+)$$ wird die Konzentration beschrieben und kann entsprechend Umgestellt werden: $$\left. K_L\ =\ c(Cl^-)\cdot c(Ag^+)\ \right| K_L=1,6\cdot 10^{-10}\frac{mol^2}{L^2}\ \wedge\ c(Cl^-)=1\frac{mol}{L}$$ $$\left. 1,6\cdot 10^{-10}\frac{mol^2}{L^2}\ =\ 1\frac{mol}{L} \cdot c(Ag^+)\ \right| :1\frac{mol}{L}$$ $$c(Ag^+)=1,6\cdot 10^{-10}\frac{mol}{L}$$ Die anschließende Iod Fällung basiert auf demselben Prinzip. Da wir nun herausgefunden haben, wie hoch die Konzentration von Silberionen in der Lösung ist, können wir nun auf dem gleichen Weg herausfinden, wie weit die Iod Fällung den Gehalt an Silberionen in der Lösung senken kann. Da mit den nun bekannten Größen gilt: $$1,5\cdot 10^{-16}\frac{mol^2}{L^2} < 0,06\frac{mol}{L}\cdot 1,6\cdot 10^{-10}\frac{mol}{L}$$
Fällt Silberiodid aus und die Konzentration pendelt sich nach dem oben beschriebenen Prinzip bei $$c(Ag^+) = \frac{1,5\cdot 10^{-16}\frac{mol^2}{L^2}}{0,06\frac{mol}{L}}=2,5\cdot 10^{-15}\frac{mol}{L}$$ ein. Die Konzentration wurde also um den Faktor $64000^{-1}$ verringert. Da aber noch eine messbare Menge an Silberionen in der Lösung ist, ist der Satzteil ,, ... um die verbliebenen Silberionen zu fällen ... ''\footnote{Skript S. 74} nicht akkurat.

\newpage
\section{7.2 Schwerlöslichkeit und ionische Zusätze}

In diesem Versuch wurde beobachtet, wie sich gleich-ionige bzw. fremd-ionige Zusätze auf die Löslichkeit von Kaliumperchlorat in einer Lösung auswirken. \\

\textsc{Versuchsdurchführung:} Nachdem die Menge an Kaliumperchlorat (vgl. Skript) berechnet wurde, wurde auf Anweisung des Assistenten die doppelte Menge verwendet. In diesem Fall 0,32g\\

\textsc{Beobachtung:}\hspace{5mm} Nach Zugabe von Kaliumchlorid-Lösung zu Rg 1 war ein weißer Ausfall am Boden des Reagenzglases zu erkennen. Bei Zugabe von Natriumchlorid-Lösung zu Rg. 2 war keinerlei Veränderung beobachtbar. Nach Hinzugabe des Kaliumperchlorat-Körnchens zu Rg. 3 ist das Körnchen in Lösung gegangen.\\
 

\textsc{Auswertung:}\hspace{8mm} Die Reaktionsgleichung für das Bilden des Niederschlags sieht wie folgt aus: $$ K^+_{(aq)} + ClO_{(aq)}^- \rightleftharpoons KClO_4 + H_2O$$ Fremd-ionige Zusätze erhöhen die Löslichkeit von schlecht löslichen Salzen, da zum einen das Volumen erhöht wird und sich somit mehr Salz lösen kann. Zum anderen gehen die fremden Ionen Interionische Wechselwirkungen mit den schon vorhandenen gleichen Ionen ein.\footnote{$http://www.chemgapedia.de/vsengine/vlu/vsc/de/ch/11/aac/vorlesung/kap_8/vlus/thermodynamik_loeslichkeitsprodukt.\\vlu/Page/vsc/de/ch/11/aac/vorlesung/kap_8/kap8_6/kap8_6e.vscml.html$} Durch gleich-ionige Zusätze nimmt die Löslichkeit des Salzes ab, da das Löslichkeitsprodukt überschritten wird und es, in diesem Fall, zu einer Ausfällung der Chlorationen kommt.\footnote{$http://www.chemgapedia.de/vsengine/vlu/vsc/de/ch/11/aac/vorlesung/kap_8/vlus/thermodynamik_loeslichkeitsprodukt\\.vlu/Page/vsc/de/ch/11/aac/vorlesung/kap_8/kap8_6/kap8_6d.vscml.html$}
 



\newpage
\section{7.3 Bildung von Säuren und Laugen}

Hier wird die These untersucht, dass Nichtmetalloxide bei Kontakt mit Wasser Säuren bilden, während die Oxide der Metalle bei Kontakt mit Wasser Laugen bilden.\\

\begin{center}
\textsc{Metalle}
\end{center}

\textsc{Versuchsdurchführung:} Siehe Skript.\\

\textsc{Beobachtung:}\hspace{5mm} Das Magnesium verbrennt in Heller weißer Flamme zu einem weißen Feststoff. Nach Benetzen des verbrannten Magnesiums mit Wasser und der darauf folgenden Probe mit pH-Papier färbt sich das Papier erst grünlich und später Tiefblau. Als das Calciumoxid der Lösung aus Wasser und universal Indikatorlösung hinzugegeben wurde, ist ein Farbumschlag von gelb zu Blau erkennbar.\\

\textsc{Auswertung:}\hspace{8mm} Hier wurde das Magnesium zu Magnesiumoxid verbrannt. Die Reaktion lässt sich in eine Oxidation und eine Reduktion wie folgt untergliedern: \\
\begin{center}
\begin{tabular}{lrcl}
$4\cdot Ox:$ & $Mg$ & $\rightleftharpoons$ & $Mg^{+2}$ + $2e^-$\\
$2\cdot Red:$ & $O_2$ + $4\cdot e^-$ & $\rightleftharpoons$ & $2\cdot O_{-2}$\\
$Gesamt:$ & $2\cdot Mg$ + $O_2$ & $\rightleftharpoons$ & $2\cdot MgO$\\
\end{tabular}
\end{center}
Bei der Reaktion mit Wasser entsteht Magnesium Hydroxid wie folgt: $$MgO\ +\ H_2O \rightleftharpoons\ Mg(OH)_2$$ Das Calcium verhält sich analog dazu: $$CaO\ +\ H2O\ \rightleftharpoons\ Ca(OH)_2$$ Die entstandenen Hydroxide sind Protonen Akzeptoren und damit der Grund,. weshalb die Indikator Lösung ein Basisches Millieu anzeigt.
\begin{center}
\textsc{Nichtmetalle}
\end{center}

\textsc{Versuchsdurchführung:} Siehe Skript.\\

\textsc{Beobachtung:}\hspace{5mm} Dass Phosphor verbrennt in einer weißen Flamme. Dass Anschließende hinzugegeben zu der verdünnten universal Indikator Lösung färbte die Lösung von Gelb in ein orange-rot. Der Schwefel verbrannte mit blauer Flamme und erzeugte in der Lösung ebenfalls einen Farbumschlag von Gelb zu orange-rot.\\

\textsc{Auswertung:}\hspace{8mm} Der Phosphor reagiert mit dem Sauerstoff zu Phosphoroxiden.\footnote{$http://www.chemgapedia.de/vsengine/vlu/vsc/de/ch/16/ac/elemente/vlu/15.vlu.html$} $$P_4\ +\ 5O_2\ \rightleftharpoons\ P_4O_{10}$$ In Verbindung mit Wasser reagiert das Phosphoroxid mit Wasser Zu Phosphorsäure: $$P_4O_{10}\ +\ 6\cdot H_2O\ \rightleftharpoons\ 4H_3PO_4$$ P
hosporsäure ist ein Protonendonor und gibt damit Protonen ab. Die Indikatorflüssigkeit reagiert auf die neue Konzentration an Protonen in der Lösung und die Lösung färbt sich Rot-Orange. Der Schwefel verhält sich analog dazu. Bei der Verbrennung unter Sauerstoffatmosphäre bilden sich verschiedene Schwefeloxide: $$nS\ +\ \frac{x}{2} O_2\ \rightleftharpoons\ nSO_x$$ Welche nun mit Wasser wieder Schwefelsäure formen. $$nSO_x\ +\ mH_2O\ \rightleftharpoons\ nH_2SO_4$$ 


\newpage
\section{7.4 Titration von Sickwasser}

Hier wurde experimentell die Konzentration von $H_2SO_4$ in Wasser ermittelt.

\textsc{Versuchsdurchführung:} Siehe Skript.\\

\textsc{Beobachtung:}\hspace{5mm} Während der Titration konnten folgende Werte ermittelt werden: 6,5ml ; 6,4ml ; 6,6ml\\

\textsc{Auswertung:}\hspace{8mm} Hier reagiert die Natronlauge mit der Schwefelsäure zu Dinatriumsulfat und Wasser. Die gesamte Reaktion sieht wie folgt aus: $$2NaOH\ +\ H_2SO_4\ \rightleftharpoons\ Na_2SO_4\ +\ 2H_2O$$ Unsere Messergebnisse liegen ,,gut'' beieinander. D.h. wir haben eine Standardabweichung von $\pm 0,1mL$ oder $\pm 1,5\%.$ Der Mittelwert liegt bei 6,5ml. Um zu errechnen, wie viel Schwefelsäure nun in der Probe vorhanden war, muss nun zuerst die Stoffmenge an NaOH errechnet werden, die verwendet wurde. Da 6,5ml im Durchschnitt verwendet wurde und die Lauge eine Konzentration von $0,1 \frac{mol}{L}$ hat ist die Stoffmenge: $0,0065L\ *\ 0,1\frac{mol}{L}\ =\ 0,00065mol$ Da Zwei NaOH Moleküle ein Schwefelsäuren Molekül neutralisieren, ist die Hälfte der Stoffmenge des verwendeten NaOH's die Stoffmenge der Schwefelsäure, Also $3,25\cdot 10^{-4}\ \pm 5\cdot 10^{-6}\ mol$. Die Dissoziationsgleichung von Schwefelsäure ist: $$H_2SO_4\ +\ 2\cdot H_2O\ \rightleftharpoons\ 2\cdot H_3O^+\ +\ SO_4^{2-}$$ dabei hat Schwefelsäure einen Pks-Wert von 1,9. Wenn nun folgende Formeln: $$Ks = 10^{-Pks}$$ $$Ks = \frac{c_{Produkt}}{C_{Edukt}}$$ verwendet und ineinander einsetzt, erhält man nach $c_{Produkt}$ umgestellt: $$10^{-1,9}\cdot\frac{0,000325mol}{0,025L}=1,636603035\cdot 10^{-4}$$ Dies ist die Konzentration von $H_3O^+$-Teilchen in der Probe. Eingesetzt in die Formel zur Ph-Wert-Berechnung ($Ph=-\log (cH_3O^+)$) errechnet sich ein Ph-Wert von $\approx 3,8$. Das ist sehr sauer und sehr schädlich für den Boden. Nach den Richtlinien des Ministeriums für Umwelt, Energie, Bauen und Klimaschutz Niedersachsen sollte das Grundwasser einen Ph-Wert im Bereich 6 - 8,5 haben, da sonst Schäden an Umwelt und Bauwerken verursacht werden können.\footnote{https://www.umwelt.niedersachsen.de/themen/wasser/grundwasser/grundwasserbericht/grundwasserbeschaffenheit/gueteparameter/grundprogramm/phwert/pH-Wert-137608.html}


\end{document}