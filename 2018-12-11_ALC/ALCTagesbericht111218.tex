\documentclass[12pt, a4paper]{article}
\usepackage{fullpage}
\usepackage{mathtools}

\makeatletter
\def\@seccntformat#1{
  \expandafter\ifx\csname c@#1\endcsname\c@section\else
  \csname the#1\endcsname\quad
  \fi}
\makeatother

\def\1{Lösung 1}
\def\2{Lösung 2}
\def\3{Lösung 3}
\def\4{Lösung 4}
\def\5{Lösung 5}
\def\6{Lösung 6}
\def\7{Lösung 7}
\def\8{Lösung 8}
\def\9{Lösung 9}

\begin{document}

\title{ALC Tagesbericht\\ vom 11.12.2018}
\author{Janosch Ehlers, Jacqueline Preis}
\maketitle

	\begin{center}
	\textsc{Chemisches Gleichgewicht, Katalyse}
	\end{center}

\section{6.1 Autokatalyse - Reaktion von Oxalsäure mit $KMnO_4$}

Hier wurde versucht einen qualitativen Nachweis einer Autokatalyse über Oxidation von Oxalat mit $KMnO_4$ herbeizuführen. Untersucht wurden dabei verschiedene Konzentrationen von Mangansulfat Lösungen in der Reaktion.

\textsc{Versuchsdurchführung:} Siehe Skript.\\

\textsc{Beobachtung:}\hspace{5mm} Für die Reagenzgläser wurden folgende Zeiten Ermittelt: \\
\begin{center}
\begin{tabular}{ccc}
Name & $MnSO_4 (in\ ml)$ & Zeit $(in\ s)$\\
\1 & 0,5 & 24,82\\
\2 & 0,4 & 22,04\\
\3 & 0,3 & 32,30\\
\4 & 0,2 & 25,94\\
\5 & 0,1 & 35,21\\
\6 & 0 & 1:13,88\\
\end{tabular}
\end{center}

\textsc{Auswertung:}\hspace{8mm} Die komplette Reaktionsgleichung lässt sich wie folgt aufstellen:\\

\begin{center}
\begin{tabular}{lrcl}
$5\cdot Ox:$ & $C_2O_4^{2-}$ & $\rightleftharpoons$ & $2CO_2$ + $2e^-$\\
$2\cdot Red:$ & $MnO_4^{-}$ + $ 5e^-$ + $8H^+$ & $\rightleftharpoons$ & $Mn^{2+}$ + $4H_2O$\\
\hline\\
\vspace{-8mm}\\
 & $5C_2O_4^{2-}$ + $2MnO_4^{-}$ + $16H^+$ & $\rightleftharpoons$ & $10CO_2$ + $2Mn^{2+}$ + $8H_2O$\\ 
\end{tabular}
\end{center} Erwartet wurde, dass mit abnehmender Konzentration an Mangansulfat-Lösung die Reaktionszeit zunimmt, da diese Reaktion von $Mn^{2+}$-Ionen katalysiert wird. Zudem wird bei der Reaktion auch der Katalysator $Mn^{2+}$ produziert. Somit wird in \6 allein die Dauer und Beschleunigung der Reaktion durch Autokatalyse gemessen, während in den übrigen Reagenzgläsern die Dauer und die Beschleunigung der Reaktion bei einer diskreten Vorkonzentration von Katalysator in der Lösung gemessen wird. Diese Tendenz ist eindeutig in den Messergebnissen zu erkennen, denn die Messung von \6 ist fast 3mal so groß wie die von \1. Dennoch schlagen Messungen wie die von \3 oder \4 stark aus der Reihe. Mögliche Gründe können der Menschliche Faktor (sowohl beim Zeitstoppen als auch beim Bestimmen, wann eine Lösung komplett entfärbt ist), Sauberkeit der Laborgläser und Reinheit der Chemikalien. Bessere Ergebnisse wären wahrscheinlich entstanden, wenn wir den Versuch mehrfach wiederholt hätten oder mehr Unterteilungen in den Mangansulfat-Konzentrationen gemacht hätten.

\newpage
\section{6.2 Qualitative Experimente zum chemischen Gleichgewicht}

In diesem Versuch sollte qualitativ das Gleichgewicht von optisch aktiven Ionen in Lösung nachgewiesen werden. Untersucht wurde dabei die Konzentration der Salze sowie der Ph-Wert.	
\begin{center}
\textsc{Teil A}
\end{center}
\textsc{Versuchsdurchführung:} Siehe Skript.\\

\textsc{Beobachtung:}\hspace{5mm} Die Farbe der angesetzten Lösung aus Lösung A und Wasser war ein Helles Rot-Braun. Die folgenden Ergebnisse in den einzelnen Reagenzgläsern können aus der folgenden Tabelle entnommen werden:\\

\begin{tabular}{ccc}
Reagenzglas & Inhalt + Lg. A + $H_2O$ & Farbe im Vergleich zur Ausgangslösung \\
\hline
Reagenzglas 1 & $KSCN$ & Leicht rötlicher\\
Reagenzglas 2 & $FeCl_3$ & 	Rötlicher\\
Reagenzglas 3 & $KSCN + FeCl_3$ & Stark Dunkelbraun\\
Reagenzglas 4 & - & Unverändert\\
\end{tabular}\\

\textsc{Auswertung:}\hspace{8mm} Der farbgebende Komplex hierbei ist hierbei das $Fe(SCN)_3$ dies wird bei allen Vier Reagenzgläsern gebildet. Die Reaktionsgleichungen können aus der Folgenden Liste entnommen werden: \\
Reagenzglas 1(Lg. A + $KSCN$ + Wasser):$$KSCN\ +\ H_2O\ \rightleftharpoons\ K^+_{(aq)}\ +\ SCN^-_{(aq)} $$ $$Fe(SCN)_3\ +\ H_2O\ \rightleftharpoons\ Fe^{3+}_{(aq)}\ +\ 3SCN^-_{(aq)} $$
Reagenzglas 2(Lg. A + $FeCl_3$ + Wasser):$$FeCl_3\ +\ H_2O\ \rightleftharpoons\ Fe^{3+}_{(aq)}\ +\ Cl^-_{(aq)}$$ $$Fe(SCN)_3\ +\ H_2O\ \rightleftharpoons\ Fe^{3+}_{(aq)}\ +\ 3SCN^-_{(aq)}$$ 
Reagenzglas 3(Lg. A + $KSCN\ +\ FeCl_3$ + Wasser):$$KSCN\ +\ H_2O\ \rightleftharpoons\ K^+_{(aq)}\ +\ SCN^-_{(aq)}$$ $$Fe(SCN)_3\ +\ H_2O\ \rightleftharpoons\ Fe^{3+}_{(aq)}\ +\ 3SCN^-_{(aq)}$$ $$FeCl_3\ +\ H_2O\ \rightleftharpoons\ Fe^{3+}_{(aq)}\ +\ Cl^-_{(aq)}$$
Reagenzglas 4(Lg. A + Wasser):$$3\cdot SCN^-_{(aq)}\ +\ Fe^{3+}\ \rightleftharpoons\ Fe(SCN)_3$$ In jedem Reagenzglas kann der Farbkomplex gebildet werden, die Reaktionen Unterscheiden sich allerdings in den Konzentrationsverhältnissen der Edukte. In Reagenzglas 1 wurde $KSCN$ hin zugegeben, dies geht in Lösung und die frei gewordenen $SCN^-$-Ionen können nun mit Freien $Fe^{3+}$ Ionen zu $Fe(SCN)_3$ reagieren. Das Massenwirkungsgesetz für diese Reaktion sieht nun wie folgt aus: $$K = \displaystyle{\frac{C_{Fe(SCN)_3}^1}{C_{SCN^-}^3 \cdot C_{Fe^{3+}}}}$$ Erhöht man nun hier die Konzentration der Edukte, so verschiebt sich das Gleichgewicht auf die Produkt Seite und es wird mehr $Fe(SCN)_3$ Salz gebildet. Bei Reagenzglas 1 wird nun $KSCN$ zur verdünnten Lösung A hinzugegeben. das $KSCN$ hat allerdings nur einen verhältnismäßig geringen Einfluss auf das Gleichgewicht, da es 3 $SCN^-$-Ionen braucht, um ein weiteres Salz zu bilden. In Reagenzglas 2 hingegen wurde $FeCl_3$ hin zugegeben. Dies hat einen stärkeren Effekt auf das Gleichgewicht und ist in der Lage Drei $SCN^-$-Ionen zu binden und in ein $Fe(SCN)_3$ Salz zu überführen. Dadurch lässt sich erklären, das Reagenzglas 2 eine deutlich intensivere Färbung aufweist, als Reagenzglas 1. Den denkbar stärksten Einfluss auf das Gleichgewicht hat natürlich die Hinzugabe aller Edukte, wie es in Reagenzglas 3 geschehen ist. Hier ist dementsprechend die stärkste Farbintensivität zu erkennen. Die Molarität von $Fe(SCN)_3$ in der Lösung A ist $\frac{1}{14}M$ Da: $$\frac{0,015l \cdot 1mol_{KSCN} + 0,005l \cdot 1mol_{FeCl_3}}{0,07l}$$ Da $3SCN^- + Fe^{3+} \rightleftharpoons Fe(SCN)_3$ kommt man auf: $$\frac{0,005mol}{0,07l}=\frac{1}{14}$$

\begin{center}
\textsc{Teil B}
\end{center}
\textsc{Versuchsdurchführung:} Hier wurde aufgrund von Fehlen der Cr(III)-Salzlösung nur der zweite Teil des Versuches durchgeführt. Außerdem wurde nach Abschließen des Versuches außerdem die Wiederholbarkeit des Versuchs getestet, indem im Nachhinein erneut $NaOH$-Lösung und $H_2SO_4$ Hinzu-gegeben wurde. \\

\textsc{Beobachtung:}\hspace{5mm} Die ursprünglich Orange-Gelbe $KCr_2O_7$-Lösung nahm nach Hinzugeben der $NaOH$-Lösung eine stark gelbe Farbe an. Das anschließende Hinzugeben der $H_2SO_4$-Lösung bewirkte, nach geringfügigem Schütteln, eine Farbänderung zur Ursprungsfarbe. Das Reagenzglas wurde merklich warm nach der Hinzugabe der Säure. Das anschließende Addendum zum Experiment (wie in der Versuchsdurchführung beschrieben) fiel Positiv aus. Nach erneutem Hinzugeben der $NaOH$-Lösung verfärbte sich die Lösung in ein starkes Gelb und das anschließende Hinzugeben von $H_2SO_4$ färbte die Lösung wieder in ein orange-gelb.\\

\textsc{Auswertung:}\hspace{8mm} Die Gleichgewichtsreaktion, die diesem Versuch zugrunde liegt, ist folgende: $$2K_2CrO_4\ +\ 2H^+\ \rightleftharpoons\ K_2Cr_2O_7\ +\ H_2O$$ Durch Änderung der Konzentration von $H^+$-Teilchen wird ein äußerer Zwang ausgeübt, welcher das Gleichgewicht dieser Reaktion beeinflusst. Durch Hinzugeben von $H_2SO_4$ wird das Gleichgewicht auf die Seite der Produkte bewegt, da das \textit{LeChatelier} Prinzip wirkt. Bei Hinzugabe von $NaOH$-Lösung wird die Konzentration von $H^+$-Teilchen stark reduziert. Und das Gleichgewicht wird auf die Seite der Edukte verlagert. Die Gleichgewichtskonstante kann nun qualitativ über die Farbe der Lösung bestimmt werden: Ist die Lösung Orange, so Ist das Gleichgewicht auf der Produktseite und $K>>1$. Ist die Lösung allerdings Gelb so ist $K<<1$ und das Gleichgewicht liegt auf der Edukt-Seite.\\ Hier ist zu beachten, Das Chrom(VI)-Ionen deutlich giftiger sind, als Chrom(III)-Ionen, da Chrom(III) sehr stabil ist, gibt es weniger Reaktionen, die es eingehen würde. Chrom(VI)-Ionen hingegen wirken deutlich stärker oxidierend und sind dabei auch noch weniger stabil. Das heißt, sie gehen mehr Reaktionen einfacher ein und können so auch mehr Schaden anrichten.

\newpage
\section{6.3 Das Cobaltchlorid-Gleichgewicht}

Dieses Experiment Soll den Einfluss von Konzentration der Edukte und die Temperatur auf eine Gleichgewichtsreaktion Nachweisen.

\begin{center}
\textsc{Teil A}
\end{center}
\textsc{Versuchsdurchführung:} Von der gegebenen Versuchsdurchführung wurde nicht abgewichen, allerdings wurde anschließend an den Versuch mit der Ergebnislösung noch einmal die Wiederholbarkeit des Versuch geprüft.\\

\textsc{Beobachtung:}\hspace{5mm}   Nach Erreichen des Farbumschlags und der Hinzugabe von $CaCL_2$ bildet sich eine blaue Farbe am oberen Ende der Phase, welche sich nach leichtem Schütteln auf die ganze Phase verteilt. Das anschließende Hinzugeben von Wasser (wie in der Versuchsdurchführung beschrieben) färbte die Lösung anschließend wieder Rosa.\\

\textsc{Auswertung:}\hspace{8mm} Das Blaue $CoCl_2$ reagiert mit Wasser zu einem Rosa Farbkomplex: $\left[Co(H_2)_6\right]^{2+}$ bei hoher $Cl^-$ Konzentration, kann das $\left[Co(H_2)_6\right]^{2+}$ zu $[CoCl_4]^{2-}$ reagieren. Die Gesamtreaktion sieht wie folgt aus: $$CoCl_{2\ (isopropanol)}\ +\ 6\cdot H_2O\ \rightleftharpoons\ \left[Co(H_2)_6\right]^{2+}\ +\ 2\cdot Cl^-$$ $$\left[Co(H_2)_6\right]^{2+}\ +\ 2\cdot Cl^-\ \rightleftharpoons\ [CoCl_4]^{2-}\ +\ 6\cdot H_2O$$ Das Massenwirkungsgesetz kann nun folgendermaßen aufgestellt werden: $$K=\frac{\left[Cl^-\right]^2\cdot \left[Co(H_2)_6]^{2+}\right]}{[H_2O]^6\cdot [CoCl_2]}$$ Wird nun, wie in der ersten Gleichung gezeigt, dem System Wasser hinzugefügt, Reagiert das Wasserlose Cobaltsalz zu $CoCl_2\cdot 6H_2O$. Bei anschließendem Hinzugeben von Chlorid-Ionen wird ein weiterer Komplex gebildet, das $[CoCl_4]$. Bei anschließender Erhöhung der Wasserkonzentration wird das Gleichgewicht gestört und nach dem Prinzip des kleinsten Zwangs stellt sich nun ein neues Gleichgewicht ein. In diesem Fall liegt das Gleichgewicht nun eher auf der Seite des rosafarbenen Komplexes und die Lösung erscheint Rosa.\\

\begin{center}
\textsc{Teil B}
\end{center}
\textsc{Versuchsdurchführung:} Siehe Skript.\\

\textsc{Beobachtung:}\hspace{5mm} Um den Farbumschlag zu erreichen, wurden 0,4ml Wasser verwendet. Anschließend hatte die Lösung eine sehr leicht bläuliche Farbe mit Violett-Rosa Schimmer. Die Raumtemperatur betrug hier $23,8^\circ C$. Nach Erhitzen auf $26^\circ C$ konnte beobachtet werden, das der Violette Schimmer stärker hervortrat. Bei $39,7^\circ C$ stellte sich eine leicht blaue Farbe ein. Ab $49^\circ C$ konnte die Lösung eindeutig als Blau beschrieben werden. Bei einer Temperatur von $70^\circ C$ hatte die Lösung eine tiefblaue Farbe. Das Experiment endete bei einer Temperatur von $90,1^\circ C$ hier hat die Lösung ein sattes königsblau aufgewiesen.

\textsc{Auswertung:}\hspace{8mm} beim Erhitzen einer im Gleichgewicht befindlichen Reaktion, fügt man dem System Energie zu. Dies zwingt das System den Zustand des Höheren Energiepotenzials anzunehmen. Dieser ist in diesem Fall die Verschiebung des Gleichgewichts zur Produktseite. Da dies die hinzugefügte Energie nutzt um diesen Zustand zu erreichen spricht man von einer Endothermen Reaktion.\footnote{https://www.colorado.edu/lab/lecture-demo-manual/equilibrium/e720-effect-temperature-coh2o62cocl42}


\newpage
\section{6.4 Gleichgewichtsreaktion von Kupferiodid}

Untersucht wurde hier eine Gleichgewichtsreaktion von Kupfersalzen zu einem Schwer löslichem Salz. Erwartet wird, dass das Gleichgewicht der Reaktion auf Seiten des Schwer löslichen Salzes liegt. \\

\textsc{Versuchsdurchführung:} Auch hier wurde der Versuch auf seine Wiederholbarkeit mit der Ergebnislösung getestet, dabei wurden Schritt c. und d., nach Absolvieren dieser, wiederholt.\\

\textsc{Beobachtung:}\hspace{5mm}Nach Bildung der Lösung aus $Wasser$, $KI$ und $CuSO_4 \cdot 5H_2O$ entstand eine leicht blaue Lösung. Das anschließende Hinzugeben von Cyclohexan bewirkte die Bildung Zweier Phasen. Die oben stehende Phase wies eine leicht violette Farbe auf, die Untere hingegen eine Grasgrüne Farbe. Das Hinzugeben von Ammoniaklösung bewirkte eine Verfärbung der unteren Phase in ein Tiefblau. Die anschließend hin zugegebene Schwefelsäure hat die ursprüngliche Verfärbung wieder hergestellt. Die anschließend folgende Prüfung auf Wiederholbarkeit, wie in der Versuchsbeschreibung erklärt ist, fiel positiv aus. Nach erneuter Hinzugabe von Ammoniak Lösung färbte sich die Lösung Blau und bei Hinzugabe von Schwefelsäure wieder grasgrün. Allerdings konnte an keinem Punkt der Reaktion das Ausfallen eines Feststoffs oder Trübung einer Lösung beobachtet werden\\

\textsc{Auswertung:}\hspace{8mm} 	Die der Theorie hinter diesem Versuch zugrundeliegenden Reaktionsgleichungen sehen wie folgt aus:\\
\vspace{-10mm}\\
\begin{center}
\begin{tabular}{lcrcl}
(1)& & $\displaystyle{2Cu_{(aq)}^{2+}}$ + $4I^-$ & $\rightleftharpoons$ & $2CuI_{(s)}$ + $I_{2\ (aq)}$\\
\vspace{-3mm}\\
(2)& Ox: & $2I^-_{(aq)}$ & $\rightleftharpoons$ & $I_{2\ (aq)}$ + $ 2e^-$\\
\vspace{-3mm}\\
(3)& Red: & $Cu_{(aq)}^{2+}$ + $e^-$ & $\rightleftharpoons$ & $Cu^+$\\
\vspace{-3mm}\\
(4)& & $I_{2\ (Wasser)}$ & $\rightleftharpoons$ & $I_{2\ (Cyclohexan)}^*$\\
\vspace{-3mm}\\
(5)& & $Cu^{2+}$ + $4NH_3$ & $\rightleftharpoons$ & $\displaystyle{\left[Cu(NH_3)_4\right]^{2+}}$\\
\vspace{-3mm}\\
(6)& & $H_2SO_4$ & $\rightleftharpoons$ & $2H^+$ + $SO_4^{2-}$\\
\vspace{-3mm}\\
(7)& & $NH_3$ + $H^+$ & $\rightleftharpoons$ & $NH_4^+$\\
\end{tabular}
\end{center}
Die Reaktionsgleichung (1) beinhaltet zwei Prozesse auf einmal: Eine Fällungsreaktion und eine Redoxreaktion. Die Redoxreaktion ist leicht zu bestimmen, denn die Produkte verursachen beim Reagieren eine Elektronen Verschiebung. So kann diese Teilreaktion in Oxidation (2) und Reduktion (3) aufgespalten werden. Zudem wird bei der Reaktion ein Salz gebildet ($CuI$ in (1) ), welches aufgrund seiner schlechten Lösungseigenschaften ausfällt. Dieser Prozess wird Fällungsreaktion genannt. \\ Der Begriff trübe Lösung wäre, bei gegebenen Anlass, nicht zutreffend. Da das Wort trüb impliziert, dass hier Feststoffe in einer Flüssigkeit suspendiert sind. Ähnlich wie bei Matsch-Wasser. Hier können alle Stoffe, die für die Trübung verantwortlich sind, durch Filterung entfernt werden. Während dies genau der Definition einer Lösung widerspricht. Treffender wären die Begriffe ,,verfärbt'' / ,,Farblos''. Dabei wird klargemacht, ob optisch aktive Bestandteile in der Lösung vorhanden sind, oder nicht. In allen anderen Fällen beschreibt man eine Suspension bzw. den Feststoff und die Flüssigkeit und keine Lösung.\\ Die Farben, die während der Reaktion beobachtet werden konnten, sind einfach zu erklären: nach Hinzugabe, der Ammoniak Lösung ist, Reaktion (5) abgelaufen. Es wurde ein Komplex ($\displaystyle{\left[Cu(NH_3)_4\right]^{2+}}$) gebildet, der für die starke Blaufärbung verantwortlich ist. Nach an säuern der Lösung durch $H_2SO_4$ wird das Gleichgewicht verschoben, da nun der Ammoniak mit den Protonen zu Ammoniumionen reagieren (7). Dadurch werden nun die Kupferionen für eine Reaktion mit dem Iodid frei, wie in (1) zusehen. Das Iod löst sich besser in Cyclohexan, als in Wasser. Aufgrund des kleinsten Zwanges liegt somit das Gleichgewicht auf der Seite des Cyclohexans, weshalb sich die obere Phase stärker Violett färbt als die untere. Da die Lösung nun wieder im Ausgangszustand ist, kann die Reaktion nun wiederholt werden. Bei der Reaktion (1) sollte nun auch noch ein Feststoff ausfallen. Dies geschah, wie in der Versuchsbeobachtung beschreiben, nicht. Grund hierfür könnten unter anderem Verunreinigungen der Chemikalien oder des Laborgeräts sein. Denkbar ist zum Beispiel die Bildung von Kupferlegierungen wie Grünspan, welche dann in einer grünen Farbe erscheinen.

\end{document}
