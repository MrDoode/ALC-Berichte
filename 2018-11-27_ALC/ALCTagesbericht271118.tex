\documentclass[11pt, a4paper]{article}
\usepackage{fullpage}

\begin{document}

\title{ALC Tagesbericht\\ vom 27.11.2018}
\author{Janosch Ehlers, Jacqueline Preis}
\maketitle

	\begin{center}	
	\textsc{Redox- Reaktionen \& chemische Energetik}
	\end{center}

\section{4.1 Einfache Redoxreaktionen}

Hier wurde eine Redoxreaktion mit dem Nachweiß für Stärke kombiniert. Im zweiten Teil wurde untersucht ob Sulfit-Ionen durch Kaliumpermanganat zu Sulfat oxidiert werden kann. Im Dritten Teil hingegen wurden die verschiedenen Reaktionen von Salpetersäure und Unedlen Metallen untersucht. 
\begin{center}
\textsc{Teil A}
\end{center}
\textsc{Versuchsdurchführung:} Siehe Skript.\\

\textsc{Beobachtung:}\hspace{5mm}
Nach hinzugabe von Kaliumiodidlösung, entstand eine farblose Lösung. Nach hinzugabe vom Chlorwasser färbte sich die Lösung Dunkel-Violett.\\

\textsc{Auswertung:}\hspace{8mm} Die hier durchgeführte reaktion war eine abwandlung der Nachweißreaktion von Stärke bzw. Iod. Es wird bei dieser Reaktion Iod gebildet, welches sich in die Helixstrukturen der Stärke einlagert. Die Redoxreaktion sieht wie folgt aus:\\
\begin{center}
\begin{tabular}{crcl}
Oxidation: & $2KI$ & $\displaystyle{\rightleftharpoons}$ & $I_2 + 2e^- + 2K^+$\\
Reduktion: & $Cl_2 + 2e^-$ & $\displaystyle{\rightleftharpoons}$ & $2Cl^-$\\
\hline
	& $Cl_{2_{(aq)}} + KI$ & $\displaystyle{\rightleftharpoons}$ & $I_2 + 2KCl$
\end{tabular}
\end{center}


\begin{center}
\textsc{Teil B}
\end{center}

\textsc{Versuchsdurchführung:} Siehe Skript.\\

\textsc{Beobachtung:}\hspace{5mm}
Die angesäuerte Permanganatlösung war stark Violett gefärbt. Nach hinzugabe von $NaSO_3$ klarte die Lösung sofort auf. Es bildete sich eine Farblose löung und eine zweite untenstehende Phase, welche der gleiche Farbe hatte wie die ausgangslösung.\\

\textsc{Auswertung:}\hspace{8mm} Das hier ist eine simple Redoxreaktion mit Kaliumpermanganat als Oxidierungsmittel. $NaSO_3$ wurde hier in dieser Reaktion zu $NaSO_4$ oxidiert. Die gesamtreaktion sieht wie folgt aus:
\begin{center}
\begin{tabular}{crcl}
Oxidation: & $NaSO_3 + 3H_2O$ & $\displaystyle{\rightleftharpoons}$ & $NaSO_4 + 2e^- + 3H_3O^+$\\
Reduktion: & $KMnO_4 + 8e^- + 8H_3O^+ $ & $\displaystyle{\rightleftharpoons}$ & $KMn + 12H_2O$\\
\hline\\
\vspace{-6mm}\\
	& $8Na_2SO_3 + 24H_2O + 2KMnO_4 + 8H_3O^+$ & $\displaystyle{\rightleftharpoons}$ & $8NaSO_4 + 2H_3O^+ + 2KMn + 12 H_2O$\\
\vspace{-2mm}\\
\hline\\
\vspace{-6mm}\\
	& $4Na_2SO_3 + 6 H_2O + KMnO_4 + 3H_3O^+$ & $\rightleftharpoons$ & $4NaSO_4 + KMn$\\
\vspace{-6mm}
\end{tabular}
\end{center}


\begin{center}
\textsc{Teil C}
\end{center}

\textsc{Versuchsdurchführung:} Siehe Skript.\\

\textsc{Beobachtung:}\hspace{5mm}
Im ersten Reagenzglas konnten folgende Reaktionen beobachtet werden:\\ Im ersten Reagenzglas konnte eine leichte Schaumbildung beobachtet werden. Nach ein paar Minuten konnte klarte die Lösung auf. Das Zweite Reagenzglas, in welches die Halbkonzentrierte Salpetersäure hinzugegeben wurde konnte eine starke Schaumbildung und die bildung von Rot-braunem Dampf beobachtet werden. Die Konzentrierte Salpetersäure im Dritten Reagenzglas hat derart stark reagiert, dass auch bei tropfenweiser hinzugabe der $HNO_3$ eine sehr starke schaumbildung enstand. Auch diese Lösung klarte nach einer Zeit auf. hier konnten danach auch keine Zink Reste mehr betrachtet werden.\\

\textsc{Auswertung:}\hspace{8mm} Die hier ablaufende reaktion von, ist eine simple redoxreaktion mit Salpetersäure als Oxidationsmittel.Die Reaktion im ersten Reagenzglas lässt sich wie folgt beeschreiben:
\begin{center}
\begin{tabular}{crcl}
Oxidation: & $Zn$ & $\displaystyle{\rightleftharpoons}$ & $Zn^+ + 2e^-$\\
Reduktion: & $2HNO_3 + 2e^-$ & $\displaystyle{\rightleftharpoons}$ & $H_2 + 2\cdot (NO_3)^-$\\
\hline
	& $2Zn + 4HNO_3$ & $\displaystyle{\rightleftharpoons}$ & $2Zn^2+ +2H_2 + 4\cdot (NO_3)^-$
\end{tabular}
\end{center}

Die Reaktionsgleichung, die der Reaktion im Zweiten Reagenzglas zugrundeliegt lässt sich wie folgt beschreiben:

\begin{center}
\begin{tabular}{crcl}
$3\cdot Oxidation$: & $Zn$ & $\displaystyle{\rightleftharpoons}$ & $Zn^+ + 2e^-$\\
$2\cdot Reduktion$: & $4HNO_3 + 3e^-$ & $\displaystyle{\rightleftharpoons}$ & $NO + 3\cdot (NO_3)^- + 2H_2O$\\
\hline
	& $3Zn + 8HNO_3$ & $\displaystyle{\rightleftharpoons}$ & $2NO + 3Zn(NO_3)_2 + 4H_2O$
\end{tabular}
\end{center}

Die dritte reaktion lässt sich wie folgendermaßen beschreiben:

\begin{center}
\begin{tabular}{crcl}
$Oxidation$: & $Zn$ & $\displaystyle{\rightleftharpoons}$ & $Zn^+ + 2e^-$\\
$2\cdot Reduktion$: & $2HNO_3 + e^-$ & $\displaystyle{\rightleftharpoons}$ & $NO_2 + NO_3^- + H_2O$\\
\hline
	& $Zn + 4HNO_3$ & $\displaystyle{\rightleftharpoons}$ & $2NO_2 + Zn(NO_3)_2 + 2H_2O$
\end{tabular}
\end{center}


\newpage
\section{4.2 Eloxieren von Aluminium}

Hier wurde untersucht wie $Al_2O_3$-Schicht und die reaktivität des zu untersuchenden Aluminiumstücks zusammenhängen.
\begin{center}
\textsc{Teil A}
\end{center}

\textsc{Versuchsdurchführung:} Siehe Skript.\\

\textsc{Beobachtung:}\hspace{5mm} Die Aluminiumfolie begann sofort nach hinzugebn zu der Fllüssigkeit, sich aufzulösen. Dabei entstand ein Gas und es bildete sich eine große Menge an rotem Niederschlag. Außerdem wurde viel Wärme erzeugt.

\textsc{Auswertung:}\hspace{8mm} 

\begin{center}
\textsc{Teil B}
\end{center}

\textsc{Versuchsdurchführung:} Siehe Skript.\\

\textsc{Beobachtung:}\hspace{5mm} \textbf{i)}Nach starten des Versuchs bildet sich eine braune Schicht auf der Aluminiumplatte. \textbf{ii)} Bei einer angesetzten Spannung von 6,0V und 0,73A konnte eine leichte Gasentwicklung an der Kathode festegestellt werden. \textbf{iii)} Hier konnte nun eine deutlich größere Gasentwicklung festgestellt werden. Bei der Kathode entstand deutlich mehr Gas als bei der Anode. Desweiteren färbte sich die Lösung leicht Grau und wurde Warm. Auf der Aluminiumplatte entstad außerdem eine Schwarze Ablagerung. \textbf{iv)} Die entstandene Ablagerung auf der Aluminiumplatte von iii) löste sich auf und es bildete sich eine Kupferfarbene Schicht auf der Aluminiumplatte, allerdings deutlich weniger als bei i). \textbf{v)} nach kratzen der Aluminiumplatte in der Lösung konnte sofort die starke Bildung einer Kupferfarbenen Schicht entlang der Kratzspuren entdeckt werden. 

\textsc{Auswertung:}\hspace{8mm} 


\newpage
\section{4.3 Modell der Opferanode}

Hier wurde untersucht wie sich die reaktivität von unedlen metallen verändert, wenn ein Stoff mit geringerem elektrochemischem Energiepotential (gemessen an der Standartwasserstoffelektrode) vorhanden ist.
 
\textsc{Versuchsdurchführung:} Siehe Skript.\\

\textsc{Beobachtung:}\hspace{5mm} Nach eintauchen des Anspitzers in die Lösung konnte nach einigen Minuten die bildung von kleinen Rostroten Punkten auf den Klingen des Anspitzers und kleine Weiße punkte auf dem Körper des selbigen entdeckt werden. Die PH-wert messung ergab einen Wert von ~6 

\textsc{Auswertung:}\hspace{8mm} 


\newpage
\section{4.4 Wärmekissen und Kältemischungen}

Hier wurde untersucht wie sich Lösungen erstellen lassen, die bestimmte wärmende bzw. kühlende Wirkungen haben.
\begin{center}
\textsc{Teil A}
\end{center}

\textsc{Versuchsdurchführung:} Siehe Skript.\\

\textsc{Beobachtung:}\hspace{5mm} Nachdem wir die Lösung, nach erreichen der Raumtemperatur, haben kristallisieren lassen, konnten wir eine Temperatur von $41^\circ C$ messen. 

\textsc{Auswertung:}\hspace{8mm} \\
\begin{tabular}{rcl}
 & + Wärme & \\
 $CH_3COONa\cdot 3H_2O + nH_2O$ & $\rightleftharpoons$ & $(CH_3COONa\cdot 3H_2O)_{(aq)} + nH_2O$\\
  & - Wärme & \\
\end{tabular}\\
Wenn Wärmekissen aus kristallisieren dann gibt das Salz in der Lösung, im Wärmekissen, die Energie frei die das Salz in Lösung hält. Erhitzt man das Wärmekissen wieder wird dem System wieder energie hizugefügt und das Salz löst sich wieder in der Flüssigkeit. Die Flüssigkeit verbleibt während das Salz kristallin vorliegt als (z.B.) Kristallwasser vor. Nach dem lösen liegt eine übersättigte Lösung vor. Hier ist mehr Salz in einer flüssigkeit gelöst als die Flüssigkeit binden kann. um in diesen Zustand zu kommen muss viel Energie aufgebracht werden. Eine übersättigte Lösung ist sehr instabil, da das Salz das bestreben hat zu kristallisieren und seine Energie abzugeben. hergestellt werden diese Lösungen indem sehr viel Salz unter zufuhr von Wärme wenig Flüssigkeit zugeführt wird und dannach auf Raumtemperatur abgekühlt wird.\\

\begin{center}
\textsc{Teil B}
\end{center}

\textsc{Versuchsdurchführung:} Hier wurde in Versuch B2) $CaCl_2\ \cdot\ 2H_2O$ verwendet anstelle von $CaCl_2\ \cdot\ 6H_2O$. In allen anderen Punkten siehe Skript.\\

\textsc{Beobachtung:}\hspace{5mm} Nach erstellen der 1. Lösung (B1) konnte eine Temperatur von $-11^\circ C$ gemessen werden. Die Zweite Lösung erreichte nach ansetzen eine Temperatur von $13^\circ C$. Die Dritte Lösung erreichte eine Temperatur von $-39,7^\circ C$. Die Raumtemperatur betrug $19^\circ C$.

\textsc{Auswertung:}\hspace{8mm} lp
Bei einem Kühlprozess wird Energie inform von Wärme ähnlich wie bei einem Druckausgleich, oder Diffusion von Ionen vom Punkt der höchsten konzentration zum Punkt der Niedrigsten Konzentration transportiert. Bei kühlung mit Eis wird das Eiserwärmt und entzieht dabei der Umgebung Wärme, welche selbst somit zum kühlmittel wird und somit ihrer Umgebung Wärme entzieht. Wärme selbst ist die Bewegungsgeschwindigkeit der Teilchen eines Stoffes.

\end{document}